\section{Units of Meaning?}
\begin{quote}
  \emph{``To many people the most obvious feature of a language is that it consists of words''} - \textcite{halliday-2004-lexicology}
\end{quote}

What is a word? The `Word' workshop, held at the international research centre for linguistic typology 12-16 August 2000 consisted of 16 presentations discussing the answer to this question. Ten of those were published in the book \citetitle{dixon02-word}\smartcite{dixon02-word}. The difficulty of the question is summarised in the book's conclusion as follows:

\begin{quote}
\emph{``The problem of the word has worried general linguists for the best part of a century. In investigating any language, one can hardly fail to make divisions between units that are word-like by at least some of the relevant criteria. But these units may be simple or both long and complex, and other criteria may establish other units. It is therefore natural to ask if ‘words’ are universal, or what properties might define them as such.''}
\end{quote}

Consider \emph{so called} filler-words like ``uh'' or ``um'', or compound forms like ``rock 'n' roll'' and ``New York''. These are examples of language elements that are sometimes treated as words and sometimes not; it is not clear where the boundary should be drawn. Even agreed upon boundaries change with time; the word ``tomorrow'', listed in Johnson's 1755 dictionary of the English language as two distinct components ``to'', and ``morrow'' yet these parts have merged and are considered a single indivisible word. Halliday \smartcite{halliday-2004-lexicology} asks ``How do we decide about sequences like lunchtime (lunch-time, lunch time), dinner-time, breakfast time? How many words in isn't, pick-me-up, CD?''.

The understanding of `word' formed within the English language cannot simply be applied beyond that scope either, \cite{dixon02-word} notes ``The idea of ‘word’ as a unit of language was developed for the familiar languages of Europe''. Languages written without spaces, like Chinese, Japanese, and Thai present one particular example of the challenges in applying the `word' concept beyond this scope: a single sentence may be divided up into different word-like units depending on the approach taken,\textcite{chen-2017-adversarial-multi} gives the following example:

\begin{align*}
&\text{The sentence "Yao Ming reaches the finals" (姚明进入总决赛)}\\
  &\text{May be divided up to yield either three words or five using the }\\
  &\text{two segmentations methods; 'Chinese Treebank' segmentation, or}\\
  &\text{'Peking University' segmentation, respectively:}\\
\\
&\texttt{姚明~~~~~进入~~~~~总决赛~}\\
&\texttt{YaoMing~~reaches~~finals}\\
&\\
&\texttt{姚~~~明~~~~进入~~~~~总~~~~~~~决赛}\\
&\texttt{Yao~~Ming~~reaches~~overall~~finals}\\
\end{align*}

Different languages present different difficulties -In the case of Arabic, expressions formed by adding consecutive suffixes to a single root word (like establish $\to$ establishment $\to$ establishmentarianism) are a more essential part of the language, to the extent that complete sentences can be expressed as single word-units. As a result, `words' in the sense of `things occurring between spaces' describe a unit of meaning larger and more complex than the term describes in English, meaning representations of Arabic words defined this way describe something quite different to the English analogue.

The term \textbf{lexical item} describes `units' of language more broadly; in Chinese they may be individual characters or groups with a conventional meaning. in Arabic they may be the set of roots and suffixes In English they may be the words, inflections, and multi-word phrases used as single parts.

