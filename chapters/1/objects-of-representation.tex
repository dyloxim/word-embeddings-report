\section{Objects of Representation}
\textbf{Lexicology} is the study of words, and \textbf{semantics} is the study of meaning. The `embedding' in `word embeddings' derives from the mathematical sense `to embed in a space' (for example, to embed a graph in a coordinate space). Word embedding algorithms are methods for embedding \emph{lexicological objects} in \emph{semantic space}, i.e., words a space where position relates to meaning somehow). This is achieved by statistical analysis of large quantities of written language (by extraction of word-tokens).

The conventions of written language, called \textbf{orthography}, make word-tokens an imperfect representation of `words' in the lexicological sense. In particular, one problem is that distinct words do not always have distinct realisations in written language. Orthographic forms with this property are called \textbf{polysemus} (poly as in \emph{many} and semus as in \emph{meaning}, from the same root as semantics). Alternatively, the distinct words sharing this same form are referred to as \textbf{homonyms} (having the same name). The words \emph{bank} as in money holding organisation and \emph{bank} as in riverside are often used to illustrate this property. Indeed, some words, like \emph{frequent} as in often, and \emph{frequent} as in attend are examples of words distinct in spoken language despite being identical in written language; such words are said to be \textbf{homographs} (same orthography, different names).

\begin{figure}[h]
 \centering
 \includesvg[width=0.9 \columnwidth]{figures/inkscape/lexicology-orthography-embedding-PATHS.svg}
 \caption{The \emph{problem of polysemy} illustrated}
\end{figure}

Embedding algorithms which generate their representations without taking this \emph{problem of polysemy} into account produce representations which conflate the multiple meanings of a word-form into a single point in the embedding space.
