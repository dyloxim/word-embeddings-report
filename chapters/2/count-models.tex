\section{Count Models}
The embedding algorithms discussed in this report represent the confluence of two key ideas: first that the meaning of a word corresponds with the contexts in which it occurs, and second that the meaning of a word can be represented by a vector of numerical values.

This first idea is called the \textbf{distributional hypothesis}, was ``widespread in linguistic theory in the 1950s'' Jurafsky (2021), with key contributions to its conception contributed by Joos (1950) \emph{Description of Language Design}, Harris (1954) \emph{Distributional Structure}, and Firth (1957) \emph{A synopsis of linguistic theory 1930-1955}.

\begin{quote}
To illustrate the distributional hypothesis, consider for example the possible meanings of ``bank'' discussed above in relation to polysemy: \emph{bank} as in the organisation, and \emph{bank} as in riverside. The first sense of the word is identified with neighboring word contexts like: account, manager, desk, etc. whereas the second sense is distinguished by context words like: river, grassy, marsh.
\end{quote}

The way these distributional relationships are inferred by analysis of a \textbf{corpus} (a collection of reference texts) and synthesised in vector form differ according to the embedding algorithm used. The `count models' developed in the 1960s and refined thereafter were the first to apply the distributional hypothesis to create word vectors, and remained the state of the art until a series of papers published by Milokov et al (2013) presented efficient methods for generating word vectors using neural networks, along with a standard software implementation $\texttt{word2vec}$ which then became ubiquitous.

We will discuss count models first as they lay the contextual foundation for the innovations of the later predict models.
\subsection{History}
The adoption of distributional perspectives within linguistic spheres in the 1950s coincided with an emerging interest in mechanical and computational methods for document indexing and information retrieval (IR); this purpose was one of the first applications considered for modern computers. The question of quantifying document `aboutness' and relatedness suggested, as viewed through the right mathematical lens, a notion of a `document space' populated by `document vectors' where relatedness could be quantified by a distance function on the space. From this the notion word vectors naturally followed. We summarise some key events in this lineage below.

In 1948 two chemists and leading advocates of punched cards, James W. Perry and G. Malcolm Dyson, petitioned IBM head, Thomas Watson, Sr to invest in research \& development on ``mechanical solutions for scientists’ specific information problems'', who selected Hans Peter Luhn for the task [cite: Rieder, B 2020].

This appointment resulted in ``a remarkable series of papers'' [cite: salton 1987] published by Luhn between 1957 \& 1959, in particular: \emph{A Statistical Approach to Mechanized Encoding and Searching of Literary Information}, Luhn (1957). The concepts discussed in this paper were prescient of future developments in the mechanical processing of documents but details of computation are omitted.

Maron \& Kuhns (1960) provided details of a potential approach to the computation of such a system in the influential \emph{On relevance, Probabilistic Indexing and Information Retrieval.} published around the same time. Crucially, the paper suggests the use of conditional probabilities to compute `closeness of index terms'. This was an application of distributional thinking to quantify word similarity, but it did not specify a vector-space conception of the problem.

This final piece was supplied by Paul Switzer in \emph{Vector Images in Document Retrieval} (1964), which references Maron \& Kuhn's paper and elaborates by suggesting the construction of a `term-term association matrix' based on co-occurence frequencies producing an `index term vector space' in which documents are then located. It is the first paper to suggest that document similarity could be determined by projecting index terms and documents into a vector space.


Details of the resulting model and the most widely adopted refinements are the focus of this section.

\subsection{Term-Document \& Term-Term Matricies}

\begin{definition}[Document Word Occurences]
  The number of times a word-token $w$ occurs in a tokenized document $W_d$ is given by $\operatorname{count}(w,W_d)$.
\end{definition}

\begin{definition}[Term Document Matrix]
  A term-document matrix $X$, (for corpus $C$ and tokenizer $T$) has columns corresponding to each corpus document, and rows corresponding to each word in the corpus vocabulary (hence having dimensions $\#V(C)\times \#C$). The matrix entry 
\end{definition}

\subsection{Term Weighting}

\subsection{Smoothing}

\subsection{Singular Value Decomposition}
