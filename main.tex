%%% Local Variables:
%%% TeX-engine: xelatex
\documentclass{ucetd}

% general
\usepackage{fontspec}
\usepackage{xeCJK}
\usepackage{titlesec}
\usepackage{subfigure,epsfig,amsfonts}
\usepackage{amsmath}
\usepackage{amssymb}
\usepackage{mathrsfs}
\usepackage{wrapfig}
\newenvironment{figurewrap}
 {%
%  \setlength{\intextsep}{0pt}% <--- Wrong!
  \setlength{\columnsep}{15pt}%
  \wrapfloat{figure}%
 }
 {\endwrapfloat}
\usepackage{graphicx}
\usepackage{tikz,pgf}
\usetikzlibrary{backgrounds}
\usepackage{booktabs}
\usepackage{wasysym}
\usepackage{multicol}
\usepackage[linesnumbered,lined,boxed,commentsnumbered]{algorithm2e}
\usepackage{capt-of}
\usepackage{caption}
\usepackage{float}
\usepackage[dvipsnames]{xcolor}
\usepackage{soul}
\usepackage{empheq}
\usepackage{enumitem}
\usepackage[many]{tcolorbox}
\usepackage[
  left=1.35in,
  textwidth=5.9in,
  includeheadfoot,
  voffset=0.5em,
  headsep=1.0em,
  headheight=1.5em,
  textheight=635pt,
  footskip=3.0em
  ]{geometry}% http://ctan.org/pkg/geometry
\setlength{\intextsep}{1.5em}
\setlength{\abovedisplayskip}{-2.0em}
\setlength{\belowdisplayskip}{-2.0em}
\setlength{\abovedisplayshortskip}{-2.5em}
\setlength{\belowdisplayshortskip}{-2.5em}
\setlength{\belowcaptionskip}{0.0em}

% require: amsmath, xcolor, soul, empheq, [many]tcolorbox
% highglighting text
\definecolor{mygreen}{rgb}{0.51, 0.94, 0.63}
\newcommand{\hll}[1]{\colorbox{mygreen}{$\displaystyle #1$}}
\tcbset{
  highlight math style={
    colback=mygreen,
    arc=0pt,
    outer arc=0pt,
    boxrule=0pt,
    top=2pt,
    bottom=2pt,
    left=2pt,
    right=2pt,
  }
}

% encircling numbers
\newcommand*\circled[1]{\tikz[baseline=(char.base)]{
    \node[shape=circle,draw,inner sep=2pt] (char) {#1};}}

% for embedding svg figures made with inkscape
\usepackage{svg}
% frames
\usepackage{framed}

\makeatletter
\newcommand\notsotiny{\@setfontsize\notsotiny\@vipt\@viipt}
\makeatother

\titlespacing{\chapter}{0pt}{-80pt}{1cm}
\titlespacing{\paragraph}{0pt}{0em}{0.8em}

% theorem environments
\usepackage{amsthm, thmtools}

% headers
\usepackage{fancyhdr}
\pagestyle{fancy}
\fancyhead[L]{\rightmark}
\fancyhead[R]{\thepage}
\renewcommand{\headrulewidth}{0pt}

%% subsubsections
\setcounter{secnumdepth}{3} % how many sectioning levels to assign numbers to
\setcounter{tocdepth}{3}    % how many sectioning levels to show in ToC
\newcounter{subsubsection}[subsection]

% references
\usepackage[colorlinks = true,%
linkcolor = black,%
urlcolor  = black,%
citecolor = black,%
anchorcolor = black]{hyperref}
\usepackage{cleveref}

% Bibliography
\usepackage[
    backend    =      biber,%
    style      = authoryear,%
    sortlocale =      en_GB,%
    url        =      false,% 
    doi        =       true,%
    eprint     =      false]{biblatex}
\addbibresource{refs.bib}

% glossary functions
% \usepackage[toc]{glossaries}
% uncomment to make glossary words bold
% \renewcommand*{\glstextformat}[1]{\textbf{#1}}

\newfontfamily\quotefont[ligatures=TeX]{Optima}
\DeclareTextFontCommand{\textquotefont}{\normalfont\quotefont}
\newcommand{\authorquote}[2]{
\begin{center}\begin{quote}
\parbox{0.25cm}{\Large{``}}{\small{\textquotefont{{#1}}}}$\,$\parbox{0.25cm}{\Large{''}} \vspace{-1.30em}\flushright{\small{\textquotefont{{#2}}}}
\end{quote}\end{center}
\vspace{-0.60em}
}
\newcommand{\basicquote}[1]{
\begin{center}\begin{quote}
\parbox{0.25cm}{\Large{``}}{\small{\textquotefont{{#1}}}}$\,$\parbox{0.25cm}{\Large{''}}
\end{quote}\end{center}
}

\usepackage[framemethod=TikZ]{mdframed}
\usetikzlibrary{calc}

\makeatletter
\newrobustcmd*\mdf@tikzbox@tfl@spare[1]{%three or four borders
    \clip(0,0)rectangle(\mdfboundingboxwidth,\mdfboundingboxheight);%
    \begin{scope}[mdfcorners]%
       \clip[preaction=mdfouterline]%
            [postaction=mdfbackground]%
            [postaction=mdfinnerline]#1;%
    \end{scope}%
    \path[mdfmiddleline,mdfcorners] ($(O|-P)-(0,0.25cm)$)--(O|-P)--(P)--($(P)-(0,0.25cm)$);
    \path[mdfmiddleline,mdfcorners] ($(P|-O)+(0,0.25cm)$)--(P|-O)--(O)--($(O)+(0,0.25cm)$);
  }%
\newrobustcmd*\changelinestyle{\let\mdf@tikzbox@tfl\mdf@tikzbox@tfl@spare}
\makeatother


% theorem styles
\declaretheoremstyle[%
headfont=\sffamily\bfseries,%
notefont=\mdseries\itshape\sffamily,%
notebraces={}{},%
headpunct=,%
bodyfont=\sffamily,%
headformat=\NAME~\NUMBER~--\NOTE\hfill\smallskip\linebreak,%
preheadhook=\begin{leftbar},%
  postfoothook=\end{leftbar},%
]{customDefinition}
\declaretheoremstyle[%
  spaceabove=2pt, spacebelow=5pt,
  headfont=\bfseries,
  postheadspace=1.4em,
  notefont=\normalfont\itshape, notebraces={(}{)},
  bodyfont=\normalfont,
]{example}

% theorem environments
\declaretheorem[name=Definition,
style=customDefinition,
numberwithi=chapter,
refname={definition,definitions},
Refname={Definition,Definitions}]{definition}
\declaretheorem[name=Note,
style=definition,
numberwithin=chapter,
refname={note,notes},
Refname={Note,Notes}]{note}
\declaretheorem[name=Example,
style=example,
numberwithin=chapter,
refname={example,examples},
Refname={Example,Examples}]{example}
\declaretheorem[name=Example,
style=example,
numberwithin=chapter,
refname={example,examples},
Refname={Example,Examples}]{example*}
\definecolor{mygrey}{rgb}{0.92, 0.92, 0.92}
\mdfdefinestyle{infobox}{%
  backgroundcolor=white,
  roundcorner=0.7pt,%
  middlelinewidth=0.7pt,%
  skipabove=10pt,%
  splittopskip=1.5em,%
  firstextra={\draw[dashed,line width=0.7pt,xshift=0.7pt] (O) -- (P|-O);},%
  secondextra={\draw[dashed,line width=0.7pt,xshift=-0.7pt] (O|-P) -- (P);},%
  middleextra={\draw[dashed,line width=0.7pt,xshift=0.7pt] (O) -- (P|-O);\draw[dashed,line width=0.7pt,xshift=-0.7pt] (O|-P) -- (P);},%
}

\newenvironment{infobox}[1]%
{\begin{mdframed}[%
    style=infobox,%
    frametitle={\mdseries\sffamily Summary: \vphantom{$frac{1}{2}$}#1},%
    skipabove=\baselineskip plus 3pt minus 1pt,%
    skipbelow=\baselineskip plus 3pt minus 1pt,%
    linewidth=0.5pt,%
    frametitlerule=true,%
    frametitlebackgroundcolor=mygrey]
  }
  {\end{mdframed}}

\surroundwithmdframed[settings=%
\changelinestyle,%
middlelinecolor=black,%
roundcorner=0.7pt,%
middlelinewidth=0.7pt,%
skipabove=10pt,%
splittopskip=1.5em,%
  firstextra={\draw[white, dashed,line width=1pt,xshift=0.7pt, pattern= on 3pt off 3pt] (O) -- (P|-O);},%
  secondextra={\draw[white, dashed,line width=1pt,xshift=-0.7pt, pattern= on 3pt off 3pt] (O|-P) -- (P);},%
  middleextra={\draw[white, dashed,line width=1pt,xshift=0.7pt, pattern= on 3pt off 3pt] (O) -- (P|-O);\draw[dashed,line width=0.7pt,xshift=-0.7pt] (O|-P) -- (P);},%
]{example}
\surroundwithmdframed[settings=%
\changelinestyle,%
middlelinecolor=black,%
roundcorner=0.7pt,%
middlelinewidth=0.7pt,%
skipabove=10pt,%
splittopskip=1.5em,%
  firstextra={\draw[white, dashed,line width=1pt,xshift=0.7pt, pattern= on 3pt off 3pt] (O) -- (P|-O);},%
  secondextra={\draw[white, dashed,line width=1pt,xshift=-0.7pt, pattern= on 3pt off 3pt] (O|-P) -- (P);},%
  middleextra={\draw[white, dashed,line width=1pt,xshift=0.7pt, pattern= on 3pt off 3pt] (O) -- (P|-O);\draw[dashed,line width=0.7pt,xshift=-0.7pt] (O|-P) -- (P);},%
]{example*}

\renewenvironment{leftbar}{%
  \def\FrameCommand{{\vrule width 3pt} \hspace{10pt}}%
  \MakeFramed {\advance\hsize-\width \FrameRestore}}%
{\endMakeFramed}


\institution{University of York}
\title{Word Embeddings}
\author{Joel Strouts}
\date{2021}
\abstract{\noindent
  %% TODO: rewrite
In this report we attempt to provide a comprehensive primer on key topics relating to word embeddings, sufficient to enable further reading of more advanced material. The reader does not need prior experience in the domain of natural language processing (NLP), but familiarity with core mathematics at an undergraduate level is assumed. We take more of a cross-disciplinary approach than purely mathematical one to reflect the primarily applied nature the subject matter. In the first chapter we summarise the key features of word embeddings: What they are, how they are created, what they are for, and other uses the theory of word embeddings outside of NLP. In the second chapter we discuss the difficulties of defining word meaning and present a framework to help distinguish between different senses of `word'. In the third chapter we describe key concepts and definitions generally assumed by the different methods for generating word embeddings. Finally in the fourth chapter we illustrate these ideas by describing the opeartion of particular class of embedding methods: count models.}

\begin{document}

\maketitle

\mainmatter{%
\tableofcontents
\chapter{Introduction}
\section{What are word embeddings?}
Word embeddings are numerical representations of words which are derived from the observed patterns of word usage in a body of example text. These representations take the form of position vectors within a \textit{Word Space}. The resulting \emph{Word Space Model} was first described by \textcite{shutze-1993-word-space} in the terms:
\basicquote{\noindent
  [The Word Space Model] uses feature vectors to represent words, but the features cannot be interpreted on their own. Vector similarity is the only information present in Word Space: semantically related words are close, unrelated words are distant.}
\begin{example}[Two Dimensional Word Space]\label{ex:2d-word-space}
  \noindent
  The word space in \autoref{fig:2d-wvs} illustrates both components of the above definition: first that related words are closer in space than less related words (the words \textit{apple} and \textit{orange} are closer in space than \textit{apple} and \textit{bicycle}), and second that the individual axis of variation are not meaningful considered in isolation from one another (there is no trait like `size'  which the $x$ or $y$ value for a word is descriptive of).
  \begin{figure}[H]
    \centering
    \begin{minipage}[t]{.4\textwidth}
      \centering
      \vspace{0.4em}
      \begin{tikzpicture}[scale=1.0]
        \draw[<->, very thick] (-2,0) -- (2,0);
        \draw[<->, very thick] (0,-2) -- (0,2);
        %% \draw[steps=.25cm,gray,very thin] (-4,-4) grid (4,4);
        \node at (-1.55 , -0.45 ) {red};
        \node at (-1.35 , -1.00 ) {orange};
        \node at ( 0.63 ,  1.50 ) {monday};
        \node at ( 1.3  ,  1.10 ) {tuesday};
        \node at ( 0.7  , -0.75 ) {bicycle};
      \end{tikzpicture}
      \caption{A two dimensional word space encoding semantic information about the five words; \emph{red, orange, monday, tuesday \& bicycle}}\label{fig:2d-wvs}
    \end{minipage}\hspace{4em}
    \begin{minipage}[t]{.4\textwidth}
      \centering
      \vspace{0.5em}
      \begin{tabular}{l r r}
        \toprule
        \multicolumn{1}{c}{\raisebox{-9pt}[0pt][0pt]{\quad word\quad}} & \multicolumn{2}{c}{features} \\
        \cmidrule(lr){2-3}
        \quad\quad & $x$ & $y$ \\
        \midrule
        red     & -1.55 & -0.45   \\
        orange  & -1.35 & -1.0  \\
        monday  &  0.63 &  1.5   \\
        tuesday &  1.3  &  1.1   \\
        bicycle &  0.7  & -0.75  \\
        \bottomrule
      \end{tabular}
      \vspace{0.42em}
      \captionof{table}{Feature vectors representing the words in \autoref{fig:2d-wvs}}\label{tab:2d-wvs}
    \end{minipage}
  \end{figure}
  \vspace{-0.5em}
\end{example}
More recent research has suggested that word spaces can in fact arrange word-representations with more order than \citeauthor{shutze-1993-word-space}'s early description of these spaces accounted for. Notably, \parencite{mikolov-etal-2013-linguistic} demonstrated their \emph{potential} for geometrically encoding semantic relationships between words (the direction in which one vector is distant from another can capture semantic information, not only the distance). \autoref{fig:word-space-geometry} illustrates the idea of geometrically encoded semantic information. Modern methodologies for assessing the quality of a word space consider the extent to which \emph{multifaceted} linguistic information about the vocabulary (not just semantic, but; phonetic, morphological, and syntactic information about the encoded words) is recoverable by the embeddings \parencite{yaghoobzadeh-schutze-2016-intrinsic, wang-2019-evaluating}.
\begin{figure}[H]
  \captionsetup{width=.91\linewidth}
  \centering
  \begin{minipage}[t]{.40\textwidth}
    \centering
    \begin{tikzpicture}[framed,scale=1.15]
      \node at ( -2  , 0) {man};
      \node at (  0  , 2) {woman};
      \draw[->, very thick, blue] (-2+0.3, 0+0.3) -- (0-0.3, 2-0.3);
      \node at (-2 , 1.5) {uncle};
      \node at ( 0 , 3.5) {aunt};
      \draw[->, very thick, blue] (-2+0.3, 1.5+0.3) -- (0-0.3, 3.5-0.3);
      \node at (-4.5 , 2 ) {king};
      \node at (-2.5 , 4 ) {queen};
      \draw[->, very thick, blue] (-4.5+0.3, 2+0.3) -- (-2.5-0.3, 4-0.3);
    \end{tikzpicture}
  \end{minipage}\hspace{3.5em}
  \begin{minipage}[t]{.40\textwidth}
    \centering
    \begin{tikzpicture}[framed,scale=1.15]
      \node at ( -2  , 0) {man};
      \node at (  0  , 2) {woman};
      \draw[->, very thick, blue] (-2+0.3, 0+0.3) -- (0-0.3, 2-0.3);
      \node at (-4.5 , 2 ) {king};
      \node at (-2.5 , 4 ) {queen};
      \draw[->, very thick, blue] (-4.5+0.3, 2+0.3) -- (-2.5-0.3, 4-0.3);
      % man -> king
      \draw[->, very thick, red] (-2-0.38, 0+0.30) -- (-4.5+0.38, 2-0.30);
      % woman -> queen
      \draw[->, very thick, red] ( 0-0.38, 2+0.30) -- (-2.5+0.38, 4-0.30);
    \end{tikzpicture}
  \end{minipage}
  \caption{An idealised illustration of how semantic \& syntactic information could be geometrically encoded by embeddings in a word space\footnotemark.}\label{fig:word-space-geometry}
\end{figure}}%
\footnotetext{We use the term \emph{idealised} here because although this figure is a reproduction of that which accompanies the famous $\overrightarrow{\text{king}}-\overrightarrow{\text{man}}+\overrightarrow{\text{woman}}=\overrightarrow{\text{queen}}$ example from~\parencite{mikolov-etal-2013-linguistic}, more recent investigations \parencite{linzen-2016-issues, rogers-etal-2017-many} have concluded that these analogical relationships are not the robust features of word space geometry which these examples suggest. Nevertheless, investigations into the geometry of word space have found some notable structure \parencite{liu-2018-visual-explor}.}\noindent
Typical word spaces, unlike the simplified form illustrated in \autoref{ex:2d-word-space}, are \emph{high-dimensional}: it is common for the dimension of a word space to be in the order of tens, hundreds, or thousands. In a relatively low-dimensional pre-trained word embedding model\footnote{Model provided by \textcite{pennington2014glove}. Available for download at \href{https://nlp.stanford.edu/projects/glove/}{(GloVe project website)}} the word orange (which is represented in the above example by the vector $\langle -1.55, 0.35\rangle$) is embedded as the 50-dimensional vector: \vspace{-1.0em}
\begin{align*}
  &\text{orange}\mapsto\\
  \text{\small{$\langle$}}
  &\texttt{\scriptsize ~-0.42783~,~~0.43089~,~-0.50351~,~~0.5776~~,~~0.097786,~~0.2608~~,~-0.68767~,~-0.31936~,~-0.25337~,~-0.37255~,}\\
  &\texttt{\scriptsize ~-0.045907,~-0.53688~,~~0.97511~,~-0.44595~,~-0.50414~,~-0.086751,~-1.0645~~,~~0.36625~,~-0.52428~,~-1.3413~~,}\\
  &\texttt{\scriptsize ~-0.2391~~,~-0.58808~,~~0.56378~,~-0.062501,~-1.7429~~,~-0.88077~,~-0.27933~,~~1.4705~~,~~0.50436~,~-0.69174~,}\\
  &\texttt{\scriptsize ~~2.0018~~,~~0.26663~,~-0.85679~,~-0.18893~,~-0.021125,~-0.055118,~-0.50337~,~-0.67157~,~~0.55502~,~-0.8009~~,}\\
  &\texttt{\scriptsize ~~0.10695~,~~0.1459~~,~-0.55588~,~-0.64971~,~~0.22046~,~~0.67415~,~-0.45119~,~-1.1462~~,~~0.16348~,~-0.62946~}
  \text{\small{$\rangle$}}\tag{\theequation}\label{eq:orange-vec}\vspace{-1.0em}
\end{align*}
High-dimensional vectors like \eqref{eq:orange-vec} cannot be plotted on an $x,y$ plane without loss of information, so we can't visualise points in a typical word space with the same approach as we used in \autoref{ex:2d-word-space}. An alternate method for visualising points in word space which still allows us to judge the similarity of vectors by their appearance, and is applicable higher dimensional vectors, is to map the magnitude of vector components onto a smooth colour gradient and then to compare the resulting colour-vectors, rather than the original numerical forms.
\begin{example}[50 dimensional Word Space]\label{ex:glove-word-space}%% todo: expand on this description
\autoref{fig:glove-word-space} applies the principle described above to visualise word vector outputs from a real word embedding model.
\begin{figure}[H]%% add key to figure
  \captionsetup{width=.91\linewidth}
 \centering
 \includesvg[width=0.91\linewidth]{figures/python/wiki_vecs.svg}
 \caption{A selection of embeddings from the same pre-trained 50-dimensional \emph{GloVe} word space model which \eqref{eq:orange-vec} was taken from.}\label{fig:glove-word-space}
 \centering
\end{figure}
\end{example}

\section{How are they created?}
Word embeddings are the output of \emph{word embedding algorithms}: Processes which take as their input large body of example language-usage and \emph{automatically produce} (i.e., without need for directions given by the practitioner regarding the processing of the data) representations of the observed words \emph{which encode a means of quantifying the semantic-relatedness of those words}.
\authorquote{What makes the word-space model unique in comparison with other geometrical models of meaning is that the space is constructed with no human intervention, and with no a priori knowledge or constraints about meaning similarities. In the word-space model, the similarities between words are automatically extracted from language data by looking at empirical evidence of real language use.}{\textcite{sahlgreen-2006-the-word-space-model}}
How exactly is this information automatically extracted from a corpora of ``real language usage'' though? \citeauthor{sahlgreen-2006-the-word-space-model} continues:
\basicquote{As data, the word-space model uses statistics about the distributional properties of words. The idea is to put words with similar distributional properties in similar regions of the word space, so that proximity reflects distributional similarity. The fundamental idea behind the use of distributional information is the so-called distributional hypothesis.}
\begin{definition}[The Distributional Hypothesis]
There are many variations of the statement of the distributional hypothesis; \citeauthor{sahlgreen-2006-the-word-space-model} opts for the more general phrasing \emph{``words with similar distributional properties have similar meanings''}, but with reference to its application in creating word spaces cites the more concrete \emph{``words with similar meanings will occur with similar neighbours if enough text material is available''} of \textcite{schutze-1995-information-retrieval} and the concise \hll{\emph{``words which are similar in meaning occur in similar contexts''}} of \textcite{rubenstein-1965-contextual-correlates}. \textcite{firth-1957-a-syn-of-lin} notably captured the essence of this hypothesis with the much quoted statement \emph{``You will know a word by the company it keeps''}.
\end{definition}\label{def:distrib-hyp}
\begin{example}[The distributional hypothesis]
  In the figure from~\autoref{ex:glove-word-space}, the vector representations of the words ``Monday'' and ``Tuesday'' are almost identical to one another. These representations were generated by analysing the contexts each word occurred in, so implied by this result is that the words ``Monday'' and ``Tuesday'' occur in very similar contexts. Consider these partial quotations\footnote{These quotations were found by searching for quotes containing the word ``Monday'' (in the case of~\eqref{eq:distrib-hyp-1} and~\eqref{eq:distrib-hyp-2}) or ``Tuesday'' (in the case of~\eqref{eq:-dstrib-hyp-3} and~\eqref{eq:distrib-hyp-4}) on the \href{https://www.goodreads.com}{goodreads} website}:
  \begin{align}
    &\text{``I'd be quite happy if I spent from Saturday night until \underline{~~~~~~~~~~} morning\dots''}\label{eq:distrib-hyp-1}\\
    &\text{``It was a \underline{~~~~~~~~~~} and they walked on a tightrope to the sun.''}\label{eq:distrib-hyp-2}\\
    &\text{``I will love you at 4:15 pm next \underline{~~~~~~~~~~}.''}\label{eq:distrib-hyp-3}\\
    &\text{``\dots the kind that blindside you at 4 pm on some idle \underline{~~~~~~~~~~}.''}\label{eq:distrib-hyp-4}
  \end{align}
  In each of these contexts it is clear that the \emph{type} of word which is missing is likely to be a day of the week but it is difficult to guess with confidence any more specifically; the words ``Monday'' and ``Tuesday'' seem to be almost equally probable guesses -\emph{their similarity of meaning is reflected in the similarity of their contexts}.
\end{example}
This approach to linguistics (distributional linguistics) was widespread in the 1950s \parencite{jurafsky21-speec}, with key contributions to the theory made by \textcite{joos-1950-description-of-language, harris-1954-distrib-struct, firth-1957-a-syn-of-lin}, and is related \citeauthor{wittgenstein53-philos}'s \parencite*{wittgenstein53-philos} ideas about `family resembelance' from the same period \parencite{turney10-from-frequen-to-meanin}.
The question of how word embeddings are created then can instead be phrased \emph{How exactly is the distributional hypothesis employed by these (word embedding) algorithms?}. Below we paraphrase the descriptions of seven categories identified by \textcite{wang-2019-evaluating} into which word embedding methods fall:
\begin{infobox}{Word Embedding Method Categorisation}
  \paragraph{*Co-occurrence Matrix}\label{cooccurrence-matrix-models} In this context we refer to either \emph{word-document} or the \emph{word-word} co-occurence matrices, though other formulations exist (\textcite{turney10-from-frequen-to-meanin} provides a good overview of approaches). The co-occurence matrix $X$ is such that the $(i, j)$ entry, $X_{ij}$ counts the number of times the word $i$ occurred in the context $j$, where $j$ could be the context of a document or the context of a near-by word.
  \paragraph{*Continuous-Bag-of-Words and skip-gram}\label{word2vec-models} Two \emph{iteration-based} methods proposed in the \texttt{word2vec} paper \parencite{mikolov13-effic-estim-word-repres-vector-space} and widely used thereafter. The first model (CBOW) predicts the center word from its surrounding context and the second (SG) predicts the surrounding context words given a center word.
  \paragraph{Neural Network Language Model}\label{nnl-model} The NNLM~\parencite{bengio-2003-a-neural-prob-lang-model} \emph{jointly learns a word vector representation and a statistical language model}, and is \emph{very computationally complex}.
  \paragraph{FastText}\label{fasttext} This method exists to address issues with other approaches which produce poor estimations for rarely used words. \emph{FastText uses subword information explicitly so embedding for rare words can still be represented well}. It is based on the skip-gram model, where each word is represented as a bag of character 
  \paragraph{$N$-gram Model}\label{n-gram-model} $n$-grams are multi-word elements with $n$ parts (for example ``a man'' is a 2-gram). $n$-gram models are methods which operate on $n$-grams in addition to, or in place of words. \textcite{zhao-etal-2017-ngram2vec} incorporates the n-gram model in various baseline embedding models such as \texttt{word2vec}, GloVe, PPMI, and SVD.
  \paragraph{Dictionary Model}\label{dictionary-model} Dictionary models learn word representations from dictionary entries as well as large unlabelled corpus, both sources of information are incorporated into one representation. \parencite{tissier-etal-2017-dict2vec} implements this method as \texttt{dict2vec}.
  \paragraph{Deep Contextualised Model}\label{deep-contextualised-model} \parencite{peters18-deep-contex-word-repres} introduced a model for \emph{deep contextualised word representations which represent complex characteristics of words and word usage across different linguistic contexts}. An Embeddings from Language Models (ElMo) representation is generated with a function that takes an entire sentence as the input.
\end{infobox}
In this report we describe algorithms for generating embeddings from the \emph{co-occurrence matrix} category (which we refer to as the ``count models''), and the \emph{continuous-bag-of-words / skip-gram} category (which we will refer to as the ``predict models'').
\section{What are they for?}
\authorquote{The task of representing words and documents is part and parcel of most, if not all, Natural Language Processing (NLP) tasks. In general, it has been found to be useful to represent them as vectors, which have an appealing, intuitive interpretation, can be the subject of useful operations (e.g. addition, subtraction, distance measures, etc) and lend themselves well to be used in many Machine Learning (ML) algorithms and strategies.}{\textcite{almeida19-word-embed}}
Although there are a variety of different word embedding algorithms, each being suited to a different niche of application, they are all united by the following traits:
\begin{infobox}{Shared Characteristics of Word Embedding Algorithms}
\paragraph{Semantic Comprehension} They provide a means of quantitatively assessing the semantic-relatedness of words;
\paragraph{Convenient Form} The semantic information is encoded by vector representations of vocabulary words;
\paragraph{Automatic Generation Process} The semantic information is \emph{``automatically extracted from language data by looking at empirical evidence of real language use''}.
\end{infobox}
The \emph{semantic comprehension} quality makes them appropriate for many tasks in natural language processing, and their \emph{convenient form} (like \citeauthor{almeida19-word-embed} notes above) in particular makes them work well as inputs to machine learning pipelines -\textcite{shutze-1993-word-space} notes: \emph{``Neural networks perform best when similarity of targets corresponds to similarity of inputs; traditional symbolic representations do not have this property''}. Finally, their \emph{automatic generation process} makes them preferable to alternative means of assessing word-relatedness which depend on extensive manual knowledge-input, like the structured lexical database: \texttt{wordnet} \parencite{miller-1990-intro-to-wordnet}.
\begin{example}[Word Clustering Using Embeddings]\label{ex:glove-word-clusters}
  \autoref{fig:glove-word-clusters} shows the result of using a point-clustering algorithm on the embeddings from \autoref{ex:glove-word-space} to identify groups of related words. The resulting clusters, identified entirely computationally, correspond with the groupings an English speaker would naturally select given the same word choices.
\end{example}
\par
\begin{figurewrap}[10]{r}{0.5\textwidth}
  \vspace{-3.5em}
  \captionsetup{width=.4\textwidth}
  \centering
  \includesvg[width=.5\textwidth]{figures/python/wiki_vecs_dendogram.svg}
  \caption{Word clusters identified by analysis of word vectors featured in \autoref{fig:glove-word-space}}\label{fig:glove-word-clusters}
\end{figurewrap}
\par
\textcite{turney10-from-frequen-to-meanin} include a thorough discussion of applications of \emph{Vector Space Models} (the class of embedding algorithms referred to as \emph{count models} in this report), listing the following examples of uses: word clustering, word classification, automatic thesaurus generation, word sense disambiguation, context sensitive spelling correction, semantic role labelling, query expansion, textual advertising \& information extraction. \autoref{ex:glove-word-clusters} illustrates the use of the embeddings from \autoref{ex:glove-word-space} to perform the first task from this list of examples: word clustering.
%% todo: list novel applications (detecting dementia etc.)
\section{Beyond word embeddings}\label{chap:beyond-word-embeddings}
Word embedding algorithms extract semantic information about language without any prior knowledge of underlying linguistic rules or significance. To these algorithms, language is just lists of discrete elements with underlying patterns to be identified. Due to this generality, word embedding algorithms can be used to encode the semantics of other types of ordered data.
Many novel applications of the embedding process to capture semantic associations outside of the written language domain have been explored. Below we list some notable embedding applications we have come across in researching for this report:
\par
\begin{infobox}{Novel Applications of the Embedding Process}
  \paragraph{Graph Embeddings} \textcite{grohe20-word2-node2} presents an overview of techniques which generalise the word embedding process to embed structured data -graphs (possibly labelled or weighted), or more generally relational structures, nodes of a large graph, or tuples appearing in a relational structure.
  \paragraph{Recommendation Systems} \textcite{grbovic-2018-real-time-personalization} demonstrate the use of user and listing embeddings to improve real-time personalisation and similar listing recommendations at Airbnb, and \textcite{wang-2018-billion-scale-commodity-embedding} employ item-graph embeddings generated from users' behaviour history to improve recommendation systems at the global e-commerce platform Alibaba.
  \paragraph{Music Embeddings} \textcite{chuan-2018-from-context-to} model functional chord relationships and harmonic associations by embedding slices of music from a large corpus music across eight genres.
  \paragraph{Customer Lifetime Value Prediction} \textcite{chamberlain17-custom-lifet-value-predic-using-embed} propose a method to generate embeddings of customers which provide daily estimates of the future value of every customer, employed by the ASOS fashion retailer.
\end{infobox}
While the focus of this document is on the use of embeddings to process language, it is our intention to present the theory of word embeddings in such a way that the underlying techniques \emph{not specific to language processing} can be understood in their own right. We intend that novel applications like those mentioned above, falling outside of the field of NLP, can be approached with confidence following this primer.

\section{Report Content \& Structure}
%% TODO: state key references (turney.. mikolov etc.)
In \autoref{chap:preliminaries} we discuss key linguistic concepts informing embedding-algorithm design: how we define `word', what is meant by `semantic relatedness' \& core concepts from distributional semantics, and finally present a formal description of data these algorithms operate on, to be used in the following chapters. In \autoref{chap:count-models} we discuss the generation of word embeddings using the vector-space-model methods following the framework laid out by \textcite{turney10-from-frequen-to-meanin}. In \autoref{chap:predict-models} we discuss the two embedding algorithms packaged by the \texttt{word2vec} software, described by \textcite{mikolov13-distr-repres-words-phras-their-compos}, and compare these methods to those already discussed. In the last chapter, \autoref{chap:practical-considerations} we address practical considerations for the application of these methods: text preprocessing, model tuning \& consideration of bias, before finally summarising observations made through the course of this research in the conclusion.
%% -------------------------- %%
%%        MAIN MATTER         %%
%% -------------------------- %%
\chapter{Words}\label{chap:words}
\newsection{What are words?}
\authorquote{To many people the most obvious feature of a language is that it consists of words}{\textcite{halliday-2004-lexicology}}
\authorquote{There have been many definitions of the word, and if any had been successful I would have given it long ago, instead of dodging the issue until now}{\textcite{matthews-1991-morph}}
\noindent
The `Word' workshop, held at the international research centre for linguistic typology 12-16 August 2000 consisted of 16 presentations discussing the answer to this question. Ten of those were published in the book \citetitle{dixon02-word}. The difficulty of the question is summarised in the book's conclusion:
\authorquote{The problem of the word has worried general linguists for the best part of a century. In investigating any language, one can hardly fail to make divisions between units that are word-like by at least some of the relevant criteria. But these units may be simple or both long and complex, and other criteria may establish other units. It is therefore natural to ask if `words' are universal, or what properties might define them as such.}{P.H. Matthews, \parencite{dixon02-word}}
\noindent
Some definitions of `word' (like \citeauthor{bloomfield-1926-a-set-of}'s \parencite*{bloomfield-1926-a-set-of} \emph{`A minimum free form is a word'}) are sufficient but not necessary, whereas others (like \citeauthor{lyons-1968-introduction}'s \parencite*{lyons-1968-introduction} \emph{`a word may be defined as the unit of a particular meaning with a particular complex of sounds capable of a particular grammatical employment'}) is a potentially necessary, but certainly not sufficient \parencite{dixon02-word}. Is it possible to conceive a watertight definition?

Our intuition is often inconsistent, making it difficult to codify -consider \emph{so called} filler-words like ``uh'' or ``um'', or compound forms like ``rock `n' roll'' and ``New York'' which are always used as whole unbroken units, what makes these \emph{not} words? Consider that the words we agree on now are subject to ever changing convention; the word ``tomorrow'' is listed in Johnson's 1755 dictionary of the English language as two distinct components ``to'', and ``morrow'' despite being indisputably considered a word \emph{to-day}. \textcite{halliday-2004-lexicology} ask ``How do we decide about sequences like lunchtime (lunch-time, lunch time), dinner-time, breakfast time? How many words in isn't, pick-me-up, CD?''.

An understanding of `word' formed within the English language cannot blindly be applied beyond that scope either, \textcite{dixon02-word} notes ``The idea of `word' as a unit of language was developed for the familiar languages of Europe''. Languages which are written without spaces, (like Chinese, Japanese, and Thai) present one particular example of the challenges in applying a local conception of `word' beyond this scope: ambiguous sentence segmentation. 
\begin{example}[Chinese Sentence Segmentation]
  \textcite{chen-2017-adversarial-multi} gives the following example of the difficulty in agreeing on `word' boundaries in Chinese: The sentence ``Yao Ming reaches the finals'' (姚明进入总决赛) May be divided up to yield either three words or five using the two segmentation methods; `Chinese Treebank' segmentation, or `Peking University' segmentation, respectively:
  \begin{align*}
    &\texttt{姚明~~~~~进入~~~~~总决赛~}\\\tag{Chinese Treebank}
    &\texttt{YaoMing~~reaches~~finals}\\
    \\
    &\texttt{姚~~~明~~~~进入~~~~~总~~~~~~~决赛}\\\tag{Peking University}
    &\texttt{Yao~~Ming~~reaches~~overall~~finals}
  \end{align*}
\end{example}
Different languages present different difficulties -In the case of Arabic, expressions formed by adding consecutive suffixes to a single root word (like the sequence \emph{establish $\to$ establishment $\to$ establishmentarianism} in English) are a more essential part of the language, such that complete sentences can be expressed as unbroken `word' units. As a result a `word' unit which is identified by unbroken character sequences occurring between spaces will, in Arabic, represent larger more complex chunks of meaning than same framework would identify for comparable sentences in English. What is the `correct' amount of information that our word-units should encode?
\section{A `word' framework}
To clarify what is meant by `word' beyond just a description of convention requires a new vocabulary: we need terms which allow us to describe the various commonalities of form and function which unify of these elements of everyday language. We present a framework due to \textcite{dixon02-word} to assist us toward this end:
\begin{infobox}{Distinguishing between uses of the term \emph{word} \parencite{dixon02-word}}
  The word `word' is used in many ways in everyday speech, and in much linguistic discourse. It is important to make certain fundamental distinctions:
  \begin{enumerate}[label=\protect\circled{\small{\arabic*}}]
    \item Between a lexeme and its varying forms;
    \item Between an orthographic word (something written between two spaces) and other types of word;
    \item Between a unit primarily defined on grammatical criteria and one primarily defined on phonological criteria.
  \end{enumerate}
\end{infobox}
Let us briefly clarify what is meant by these terms before discussing their relevance to word embeddings:
\begin{definition}[Lexemes and inflected forms]
  Lexemes are distinct \emph{root forms} which make up the \emph{lexicon} of a language (they are the elements of which dictionaries generally constitute), and serve to group so-called \emph{inflected forms} which have a meaning related to their root. Lexemes are the object being referred to as `word' in statements like ``look, looks, looked and looking are forms of the same word''. The same expression could also be phrased: ``the lexeme \emph{look} is realised as the word-forms: \emph{look, looks, looked \& looking}''.
\end{definition}
\begin{example}[Lexemes and inflected forms]\label{ex:lexemes}
  The example given above of contrasting the lexeme ``look'' to its inflected forms is presented here in a table:
  \begin{table}[H]
    \centering
    \begin{tabular}{l l l l}
      \toprule
      root or underlying form & inflected forms & grammatical function\\
      \midrule
      look                    & look            & present, non-3sg subject\\
                              & looks           & present, 3sg subject\\
                              & looked          & past\\
                              & looking         & participle\\
      \bottomrule
    \end{tabular}
    \caption{An comparison between the lexeme for the English verb \emph{look} and its inflected word-form realisations}
  \end{table}
\end{example}
\vspace{1em}
\begin{definition}[Orthographic words]
  Orthographic words are distinct elements of \emph{orthography} (written language), most often defined as \emph{something written between two spaces}.
\end{definition}
\begin{example}[Orthographic words]\label{ex:orthographic-words}
  Orthographic words entirely reflect conventions of writing, so the form ``cannot'' is one orthographic word whereas the form ``must not'' (despite having the same apparent rules of usage) is two orthographic words.
\end{example}
\begin{definition}[Grammatical and phonological words]
  This term describes the language-parts which result from applying a grammatical framework to recognise regular word-like elements in observed language. A grammatical framework is one concerned with the rules governing the arrangement of \emph{morphemes} (the smallest distinct units of meaning)
  \textcite{dixon02-word} provides the following criteria for identifying grammatical words:
  A \textbf{grammatical word} consists of a number of grammatical elements which;
  \begin{enumerate}[label=(\alph*)]
  \item always occur together, rather than scattered through the clause (the criterion of cohesiveness);
  \item occur in a fixed order
  \item have a conventionalised coherence and meaning
  \end{enumerate}
  \citeauthor{dixon02-word} also provide a definition of \emph{phonological word} in the same vein which we omit here since phonological words are less directly related to the embedding problem\footnotemark.
\end{definition}
\footnotetext{ Whereas grammatical words are concerned with elements of a grammatical structure with a regular form, phonological words are concerned with elements of a phonological structure (consisting of phonemes: the smallest distinct units of vocalised language) with a regular form. Since word embeddings deal with written language the vocal realisation of language is of less significance.}

\begin{example}[Grammatical words]
  Grammatical words conform less to our conventional notion of `word' but are more well behaved as they derive from a static criteria; The inflected forms of the lexeme \emph{look} in \autoref{ex:lexemes} (\emph{look, looks, looked, looking}) are all examples of grammatical words, but so are the forms ``New York'', ``Rock `n' Roll'', ``dinner-time'', ``breakfast time'', ``isn't'', ``pick-me-up'', and ``CD'' (which we wondered about the proper place of at the start of this chapter). Since the criteria for identifying grammatical words takes meaning into account, different words with the same word-form (like \emph{bank} (as in river) and \emph{bank} (as in financial institution)) are also considered distinct grammatical words.
\end{example}
We highlight these distinctions because without this terminology it is easy to overlook an implicit \emph{word}$=$\emph{orthographic word} assumption made when working with word embeddings. This assumption is not uncommon, as \citeauthor{dixon02-word} notes:
\basicquote{In many language communities a word is thought of as having (semantic, grammatical and phonological) unity and, in writing, words are conventionally separated by spaces}
\par
\begin{figurewrap}[13]{r}{0.6\textwidth}
  \vspace{-2.5em}
  \captionsetup{width=.55\textwidth}
  \centering
  \includesvg[width=.60\textwidth]{figures/inkscape/polysemy-figure.svg}
  \caption{The \emph{problem of polysemy} illustrated: Which sense of the orthographic words ``bank'' or ``rock'' should their embeddings encode? and what about the embedding of the grammatical word ``rock `n' roll'', not present in the orthographic words?}\label{fig:polysemy}
\end{figurewrap}
\par
but it is worth discussing explicitly since our assumptions will be encoded by our embedding algorithms. The issues relating most to the generation of word embeddings regard \emph{semantic} unity -if a word does not have semantic unity (i.e., orthographic word and semantic meaning are not one-to-one) then it is said to be \emph{polysemous} (\emph{poly} [many] \emph{semous$\rightarrow$semantic} [meanings]) and the resulting problem w.r.t. word embeddings is called \emph{the problem of polysemy}. A related problem is that of representing grammatical words which consist of multiple orthographic-word parts -these are considered as separate parts and not as a whole.
Ideally word embeddings algorithms would describe a bijective map between \emph{grammatical words} and points in word space. Since in actual fact word embedding algorithms operate on \emph{orthographic words}, the idiosyncrasies of the map (called writing) between grammatical words and orthographic words (illustrated in \autoref{fig:polysemy}) is inherited by the resulting representations. The surjectivity of this map creates the problem of polysemy and the one-to-many nature of the map means multi-part grammatical words are misrepresented too.
\citeauthor{halliday-2004-lexicology} discuss some of the pitfalls of seeking \emph{fixed, representative} meanings of words in natural language in the following excerpt:
\basicquote{As users of language we know that someone's mention of a recent television programme about big cats in Africa implies a different meaning of cat from a reference to the number of stray cats in the city of New York. And if someone talks about 'letting the cat out of the bag' or 'setting the cat among the pigeons', we know that the meaning has to be taken from the whole expression, not from a word-by-word reading of Felis catus jumping out of a bag or chasing Columbidae. Any good dictionary recognises this by such strategies as listing different senses of a word, giving examples of usage, and treating certain combinations of words (such as idioms) as lexical units. But it is important to recognise that this contextualisation of meaning is in the very nature of language and not some unfortunate deviation from an ideal situation in which every word of the language always makes exactly the same semantic contribution to any utterance or discourse. \textbf{For reasons such as these, we should be cautious about the view that words have a basic or core meaning, surrounded by peripheral or subsidiary meaning(s)}}
Clearly these qualities of language, particularly in its written form, present some difficulties for embedding algorithms. Despite this however, even the earliest embedding methods, which did not take measures to address these concerns, reported that they \emph{did} capture semantic information that was useful and interesting \parencite{deerwester-1990-indexing-by-lsa}.

The embedding methods we present in this report (elementary count models based on document-term or word-context matrices) do not address the problem of polysemy or the many-to-one problem of multi-part grammatical words, but with the information presented here the reader should: {\small\circled{1}} know the compromises made by models not accounting for these factors, and {\small\circled{2}} have the context and prerequisites upon finishing this report to understand methods which \emph{do} address the issues. In particular the \nameref{deep-contextualised-model} addresses polysemy concerns, and embeddings operating on n-grams (\nameref{n-gram-model}) address the many-to-one problem.

% ++ words: what are they?
\chapter{Preliminaries}\label{chap:preliminaries}
\section{What are words?}
\authorquote{To many people the most obvious feature of a language is that it consists of words}{\textcite{halliday-2004-lexicology}}
\authorquote{There have been many definitions of the word, and if any had been successful I would have given it long ago, instead of dodging the issue until now}{\textcite{matthews-1991-morph}}
\noindent
The `Word' workshop, held at the international research centre for linguistic typology 12-16 August 2000 consisted of 16 presentations discussing the answer to this question. Ten of those were published in the book \citetitle{dixon02-word}. The difficulty of the question is summarised in the book's conclusion as follows:
\authorquote{The problem of the word has worried general linguists for the best part of a century. In investigating any language, one can hardly fail to make divisions between units that are word-like by at least some of the relevant criteria. But these units may be simple or both long and complex, and other criteria may establish other units. It is therefore natural to ask if ‘words’ are universal, or what properties might define them as such.}{P.H. Matthews, \parencite{dixon02-word}}

Some definitions of 'word' (like \citeauthor{bloomfield-1926-a-set-of}'s \parencite*{bloomfield-1926-a-set-of} \emph{`A minimum free form is a word'}) are sufficient but not necessary, wheras others (like \citeauthor{lyons-1968-introduction}'s \parencite*{lyons-1968-introduction} \emph{`a word may be defined as the unit of a particular meaning with a particular complex of sounds capable of a particular grammatical employment'}) is a potentially necessary but certainly not sufficient criterea. Is it possible to present a watertight definition?
Our word-intuition is not watertight either, consider \emph{so called} filler-words like ``uh'' or ``um'', or compound forms like ``rock 'n' roll'' and ``New York'' always used as complete units. These are examples of language elements that are sometimes treated as words and sometimes not. The boundaries we agree on now conform to ever changing conventions; the word ``tomorrow'' is listed in Johnson's 1755 dictionary of the English language as two distinct components ``to'', and ``morrow'' despite being indesputeably a word \emph{to-day}. \textcite{halliday-2004-lexicology} asks ``How do we decide about sequences like lunchtime (lunch-time, lunch time), dinner-time, breakfast time? How many words in isn't, pick-me-up, CD?''.
An understanding of `word' formed within the English language cannot blindly be applied beyond that scope either, \cite{dixon02-word} notes ``The idea of ‘word’ as a unit of language was developed for the familiar languages of Europe''. Languages which are written without spaces, (like Chinese, Japanese, and Thai) present one particular example of the challenges in applying a local conception of `word' beyond this scope. 
\begin{example}[Chinese Sentence Segmentation]
  \textcite{chen-2017-adversarial-multi} gives the following example of the difficulty in agreeing on `word' boundaries in Chinese: The sentence "Yao Ming reaches the finals" (姚明进入总决赛) May be divided up to yield either three words or five using the two segmentations methods; 'Chinese Treebank' segmentation, or 'Peking University' segmentation, respectively:
  \begin{align*}
    &\texttt{姚明~~~~~进入~~~~~总决赛~}\\
    &\texttt{YaoMing~~reaches~~finals}\\
    \\
    &\texttt{姚~~~明~~~~进入~~~~~总~~~~~~~决赛}\\
    &\texttt{Yao~~Ming~~reaches~~overall~~finals}
  \end{align*}
\end{example}
Different languages present different difficulties -In the case of Arabic, expressions formed by adding consecutive suffixes to a single root word (like the sequence \emph{establish $\to$ establishment $\to$ establishmentarianism} in English) are a more essential part of the language, such that complete sentences can be expressed as unbroken `word' units. As a result the `word' unit (identified by unbroken character sequences occuring between spaces) in Arabic will represent larger more complex chunks of meaning than same framework would identify for comparable sentences in English. What is the `correct' amount of information that our word-units should signify?

\section{Tokens and Types}
\authorquote{The limits of my language means the limits of my world}{\textcite{wittgenstein-1922-tract}}
Contains 11 word-tokens, and 7 word-classes (assuming the sentence is tokenized by identifying word boundaries at both ends and each blank space).
\textbf{Lexicology} is the study of words, and \textbf{semantics} is the study of meaning. The `embedding' in `word embeddings' derives from the mathematical sense `to embed in a space' (for example, to embed a graph in a coordinate space). Word embedding algorithms are methods for embedding \emph{lexicological objects} in \emph{semantic space}, i.e., words a space where position relates to meaning somehow). This is achieved by statistical analysis of large quantities of written language (by extraction of word-tokens).

The conventions of written language, called \textbf{orthography}, make word-tokens an imperfect representation of `words' in the lexicological sense. In particular, one problem is that distinct words do not always have distinct realisations in written language. Orthographic forms with this property are called \textbf{polysemus} (poly as in \emph{many} and semus as in \emph{meaning}, from the same root as semantics). Alternatively, the distinct words sharing this same form are referred to as \textbf{homonyms} (having the same name). The words \emph{bank} as in money holding organisation and \emph{bank} as in riverside are often used to illustrate this property. Indeed, some words, like \emph{frequent} as in often, and \emph{frequent} as in attend are examples of words distinct in spoken language despite being identical in written language; such words are said to be \textbf{homographs} (same orthography, different names).

\begin{figure}[h]
 \centering
 \includesvg[width=0.9 \columnwidth]{figures/inkscape/lexicology-orthography-embedding-PATHS.svg}
 \caption{The \emph{problem of polysemy} illustrated}
\end{figure}

Embedding algorithms which generate their representations without taking this \emph{problem of polysemy} into account produce representations which conflate the multiple meanings of a word-form into a single point in the embedding space.

\section{Semantic relatedness}
% syntagmatic/ paradigmatic whatever
\section{Distributional analysis}
% zipfs law
% bayes etc.
\section{Terminology \& Definitions}
We begin by defining terms already encountered in mathematical terms.

\begin{definition}[Orthography]\label{def:orthography}
  An orthography is a finite set of symbols.
\end{definition}

\begin{example}\label{ex:orthographies}
  The sets:
  \begin{align}
    &\mathscr{O}_L=\{c\mid\text{$c$ is a symbol that can be copied and pasted the pdf of this report}\}\label{eq:orth-doc}\\
    &\mathscr{O}_P=\{c\mid\text{$c$ is the inscription of a regular polygon with eight sides or less}\}\label{eq:orth-shape}\\
    &\mathscr{O}_K=\{c\mid\text{$c$ is a typable character marked on \autoref{fig:uk-keyboard}}\}\label{eq:orth-keyboard}
  \end{align}
  are all examples of orthographies.
\end{example}
\vspace{6pt}

\begin{figure}[ht]
 \centering
 \includesvg[width=0.9 \columnwidth]{figures/inkscape/uk-keyboard.svg}
 \caption{A standard uk keyboard layout}
 \label{fig:uk-keyboard}
\end{figure}

Note, the term `orthography' in linguistics refers to the conventions of written language in general and is distinct from the use of `an orthography' we use here.

\begin{definition}[Document]
  A document $d$ is a finite sequence $c_0,c_1,\dots,c_n$ ($c\in\mathscr{O}$), where $\mathscr{O}$ is an orthography.
\end{definition}

\begin{example}[This Report]\label{ex:doc-report}
    The sequence of symbols obtained by copy pasting this entire report into a text editor (ordered by starting with the cursor at the top of the file and moving the cursor with the arrow keys one position to the right at a time -including newlines) is a document using the orthography \ref{eq:orth-doc}.
\end{example}
\vspace{6pt}

\begin{example}[Shapes Document]\label{ex:doc-shapes}
    The sequence of shapes: \emph{$\triangle\triangle\square\triangle\triangle\pentagon$} is a document using the orthograpy \ref{eq:orth-shape}.
\end{example}
\vspace{6pt}

\begin{example}[Plato Documents]\label{ex:doc-plato}
    Any of the works of Plato available at the \href{https://www.gutenberg.org/ebooks/author/93}{Project Gutenberg} website, copied and pasted from their plain text formats and using the same procedure described above, are documents using the orthography \ref{eq:orth-keyboard}.
\end{example}

\begin{definition}[Word-Token]
  A word token $w$ has the same attributes as a document; it is a finite sequence $c_0,c_1,\dots,c_n$ ($c\in\mathscr{O}$), where $\mathscr{O}$ is an orthography. We only refer to sequences of symbols as word tokens when they are the output of a tokenizer, however.
\end{definition}

\begin{definition}[Tokenizer]
   A tokenizer $T$ is a function that maps a document to a sequence of word tokens; $T(d)=w_1,w_2,\dots,w_n$. The output is refered to as the 'tokenized' form of the document. We use the shorthand $W_d$ to refer to the tokenized document $T(d)$ when it is clear which tokenization has been used.\footnote{Here we consider tokenization in an abstract sense essentially saying ``A tokenizer is a function that extracts tokens''. A practical discussion of tokenization and text-preprocessing is found in \autoref{sec:preprocessing}}.
\end{definition}

\begin{example}[This Report Tokenized]\label{ex:doc-report-t}
  The document from \autoref{ex:doc-report} if tokenized by treating everything occuring between two spaces (or between a space and the start or end of a line) yields a sequence of word tokens beginning \emph{``THE'' ``UNIVERSITY'' ``OF''}.
\end{example}
\vspace{6pt}

\begin{example}[Shapes Document Tokenized]\label{ex:doc-shapes-t}
  The document from \autoref{ex:doc-shapes} if tokenized using the function:
  \begin{align*}
    &T(d):\quad \texttt{Group repeated symbols in the sequence $d$ into word-tokens}\\
    &\hphantom{T(d):\ }\texttt{(in the order of occurrence).}
  \end{align*}
  produces the output: \emph{``$\triangle$ $\triangle$'', ``$\square$'', ``$\triangle$ $\triangle$'', ``$\pentagon$''}
\end{example}
\vspace{6pt}

\begin{example}[Simple Text Tokenizer]\label{ex:t-simple}
  Given a document $d$ using the orthography $\mathscr{O}_K$ from \autoref{ex:orthographies} apply the following algorithm:
  \begin{algorithm}
    \caption{Simple Text Tokenizer}
    \SetKwData{Tokens}{tokens}
    \SetKwData{NextToken}{next\_token}
    \SetKwData{Doc}{document}
    \SetKwData{Char}{character}
    \SetKwFunction{Append}{append}
    \SetKwFunction{List}{list}
    \SetKwFunction{ToLower}{convert\_to\_lowercase}
    \SetKwInOut{Input}{input}\SetKwInOut{Output}{output}

    \Input{A list of characters: \Doc}
    \Output{A list of word tokens: \Tokens}
    \BlankLine

    \Tokens $\leftarrow$ \List{}\;
    \NextToken $\leftarrow$ \List{}\;
    \For{\Char in \Doc}{
      \uIf{\Char is alphanumeric}{
        \NextToken$\leftarrow$ \Append{\NextToken, \Char}\;
      }
      \ElseIf{character is a space or a newline}{
        \If{\NextToken is not empty}{
          \NextToken$\leftarrow$\ToLower{\NextToken}\;
          \Tokens$\leftarrow$ \Append{\Tokens, \NextToken}\;
          \NextToken$\leftarrow$ \List{}\;
        }
      }
    }
  \end{algorithm}
\end{example}

\begin{definition}[Corpus]
  A corpus $C$ is a finite set of documents $\{d_1, d_2,\dots,d_n\}$.
\end{definition}

\begin{example}[Plato Corpus]\label{ex:corp-plato}
  All the of Plato works referred to in \autoref{ex:doc-plato}, if converted to documents using the same procedure described there, considered together form a corpus of documents. We will refer to this corpus as $C_p$.
\end{example}

\begin{definition}[Vocabulary]
  The vocabulary of a tokenized document $W_d$ is the set of unique word tokens in the tokenization of that document; $V(W_d)=\{w\mid w\in\, \text{the sequence $W_d$}\}$. The vocabulary of a corpus is the union of the vocabularies of each document in that corpus; $V(C)=V(d_1)\cup V(d_2)\cup \dots\cup V(d_n)$.
\end{definition}

\begin{example}[Shapes Document Vocabulary]
  Using the same tokenization as before, the document from \autoref{ex:doc-shapes} has the vocabulary:
  \begin{align*}
    &V(\text{\emph{``$\triangle$ $\triangle$'', ``$\square$'', ``$\triangle$ $\triangle$'', ``$\pentagon$''}})\\
    &\quad=\{\text{\emph{``$\triangle$ $\triangle$''}},\, \text{\emph{``$\square$''}},\, \text{\emph{``$\pentagon$''}}\}
    \end{align*}
\end{example}
\vspace{6pt}

\begin{definition}[$\#$ Operator]
  The $\#$ symbol is used to denote the number of elements of the object it precedes. Before sequences it counts the number of elements in the sequence: $\#W_d=n$ (for $W_d=w_1,\dots,w_n$) \footnote{This is like the 'word count' feature many text editors provide}. Before sets it is a shorthand for the size of that set: $\#V(d)=|V(d)|=\text{the number of words in the vocabulary $V(d)$}$. Before a corpus it counts the number of documents in that corpus.
\end{definition}

\begin{example}[Shapes Document Sizes]
  The word count of the tokenized shapes document $$W_d=\text{\emph{``$\triangle$ $\triangle$'', ``$\square$'', ``$\triangle$ $\triangle$'', ``$\pentagon$''}}$$ is $\#W_d=4$ and the vocabulary size is $\#V(W_d)=3$.
\end{example}
\vspace{6pt}

\begin{example}[Plato Document Sizes]
  Word token counts and vocabulary sizes for a selection Plato documents from~\ref{ex:doc-plato}, tokenized using the simple method: \footnote{the code used to create these examples and others examples on the same corpus is available in the appendix} have 
  \vspace{6pt}
  \begin{table}[h]
    \centering
    \begin{tabular}{c r r r r r r}
      \toprule
      \multicolumn{1}{c}{Attribute$\quad$} &
      \multicolumn{6}{c}{Document ($d$)} \\

      \cmidrule(lr){2-7}
      &
      $\texttt{laws}$ &
      $\texttt{republic}$ &
      $\texttt{cratylus}$ &
      $\texttt{meno}$ &
      $\texttt{apology}$ &
      $\texttt{crito}$ \\
      \midrule
      $\#W_d$ & 140,406 & 118,283 & 23,883 & 12,716 & 11,392 & 5,332\\
      $\#V(W_d)$ & 8100 & 8,105 & 2,963 & 1,493 & 1,789 & 1,030\\
      \bottomrule
    \end{tabular}
    \caption{Number of Word Tokens \& Vocabulary Sizes of a selection of Documents in the Plato Corpus (\autoref{ex:corp-plato})}
  \end{table}
\end{example}
\vspace{6pt}

\begin{example}[Plato Corpus Size]
  The complete Plato corpus (\ref{ex:corp-plato}) consists of 25 documents and, each tokenized the same way as before, has the word count $\sum_{d\in C_P}\#W_d=670,936$ and the vocabulary size $\#V(C_P)=19,432$.
\end{example}
%%% Local Variables:
%%% mode: latex
%%% TeX-master: "../main"
%%% End:

% ++ semantic relatedness (syntagmatic.. paradigmatic .. etc.)
% ++ distributive properties??
% ++++ zipfs law
% ++++ bayes? whatever
% ++ measures of similarity
% ++ terminology/definitions
\chapter{Count Models}\label{chap:count-models}
So-called \emph{count models} were the first methods investigated for creating word embeddings. They emerged from work in the field of information retrieval (IR). In particular, influential contributions \parencite{dublin-2004-the-most-influential} were made by \citeauthor{salton-1975-a-vector-space-model}'s \parencite*{salton-1975-a-vector-space-model} work on the SMART information retrieval system which ``pionered many of the concepts that are used in modern search engines'' \parencite{turney10-from-frequen-to-meanin}. Initally word-similarity was not the focus of the developed models, but instead \emph{document similarity} was (for matching search queries to relevant/related material), however \textcite{deerwester-1990-indexing-by-lsa} observed that by looking at row vectors rather than column vectors in a document-term matrix the objects described were \emph{word vectors}.


Co-occurrence matricies, the heart of \emph{count models} do what they say: they count instances of co-occurrence. They can be thought of like big tally tables.

We use the framework described by \textcite{turney10-from-frequen-to-meanin} to group ``count models'' according to the structure (choice of rows and columns) of the co-occurence matrix they create. In the case of document-term matricies which we discuss first, the rows represent documents and the columns represent terms in the vocabulary of the document corpus. In the case of word-context matricies the rows represent target words (occuring at the center of a context-window), and the columns represent context words (occuring in a target word's context-window).

\section{Document-term Matricies}
\begin{definition}[Document-term co-occurrence count]
  In the context of document-term matricies, the function $f(w,W_d)$ indicates the number of times the token $w$ appears in a tokenized document $W_d$.
\end{definition}

\begin{definition}[Document-term matricies]
  Given a corpus of documents $C=d_1,d_2,\dots,d_n$, and a tokenizer $T$, a term document matrix $\bf{X}$ is given by:
  \begin{equation}
    \bf{X}=
  \begin{bmatrix}
    f(w_1, W_{d_1}) & f(w_1, W_{d_2}) & \dots  & f(w_1, W_{d_n}) \\
    f(w_2, W_{d_1}) & f(w_2, W_{d_2}) & \dots  & f(w_2, W_{d_n}) \\
    \vdots        & \vdots        & \ddots & \vdots          \\
    f(w_m, W_{d_1}) & f(w_m, W_{d_2}) & \dots  & f(w_m, W_{d_n}) \\
  \end{bmatrix}
  \end{equation}
  where the $w_i$ terms are from the vocabulary of $C$, $V(C)=w_1,w_2,\dots,w_m$.
  The number of times the $i^{\text{th}}$ word appears in the $j^{\text{th}}$ document then, is given by $\bf{X}_{ij}$.
\end{definition}

\begin{example}[Plato Document Vectors]
  A document-term co-occurrence matrix has document-vectors for columns and word-vectors for rows. Here we present a section of a document-term matrix for the tokenized Plato corpus, and show in \autoref{fig:plato-docs} how these counts can be used to position document-vectors in a term-space.
  \begin{center}
  \captionsetup{width=.91\linewidth}
    \begin{tabular}{c r r r r r r}
      \toprule
      {\multicolumn{1}{c}{\raisebox{-11pt}{Term}}$\quad$} &
      \multicolumn{6}{c}{Document ($d$)} \\

      \cmidrule(lr){2-7}
      &
      $\texttt{laws}$ &
      $\texttt{republic}$ &
      $\texttt{cratylus}$ &
      $\texttt{meno}$ &
      $\texttt{apology}$ &
      $\texttt{crito}$ \\
      \midrule
      \emph{law}      & 436   & 54    & 2    & 0   & 9   & 5\\
      \emph{virtue}   & 127   & 81    & 6    & 142 & 10  & 4\\
      \emph{the}      & 9,296 & 7,048 & 1498 & 455 & 493 & 239\\
      \emph{earth}    & 24    & 29    & 14   & 0   & 5   & 0\\
      \emph{god}      & 119   & 59    & 22   & 3   & 25  & 1\\
      \emph{socrates} & 0     & 68    & 65   & 57  & 18  & 28\\
      \bottomrule
    \end{tabular}
    \captionof{table}{A document-term co-occurrence matrix for selected documents from the Plato corpus and selected terms.}\label{tab:plato-doc-term}
    \end{center}
\end{example}
\par
\begin{figure}[H]
\begin{center}
\begin{tikzpicture}[scale=0.45]
  \draw[->, very thick] (0,0) -- (14,0);
  \node at (7,-1) {\large{\emph{Socrates}}};
  \draw[->, very thick] (0,0) -- (0,14);
  \node at (-1,7) [rotate=90]{\large{\emph{God}}};
  \draw[->, very thick] (0,0) -- (0 , 11.9) node[above right]{\large{\texttt{laws}}};
  \draw[->, very thick] (0,0) -- (6.8 , 5.9) node[above right]{\large{\texttt{republic}}};
  \draw[->, very thick] (0,0) -- (6.5 , 2.2) node[above right]{\large{\texttt{cratylus}}};
  \draw[->, very thick] (0,0) -- (5.7 ,  0.3) node[above right]{\large{\texttt{meno}}};
  \draw[->, very thick] (0,0) -- (1.8 , 2.5) node[above right]{\large{\texttt{apology}}};
  \draw[->, very thick] (0,0) -- (2.8 ,  0.1) node[above right]{\large{\texttt{crito}}};
\end{tikzpicture}
\caption{Works of Plato arranged in a two dimensional term-space according to the number of times the terms ``God'' and ``Socrates'' occured in their tokenized forms.}\label{fig:plato-docs}
\end{center}
\end{figure}
The original purpose of these document-term matricies was to help relating the content of a document to queries about certain information. So, using our table above, if a query about Plato's writings contained the word ``God'' it would be more readily linked to \texttt{laws} and \texttt{apology}, wheras if the query contained ``Socrates'' the link would be with \texttt{crito} and \texttt{meno} instead.

\textcite{deerwester-1990-indexing-by-lsa} observed that this same procedure could be used to measure word-similarity, however ``a document is not necessarily the optimal length of text for measuring word similarity''. So the concept of a \emph{word-context} matrix is concieved.

\begin{definition}[Context window]
  given a token $w_i$ from a tokenized document $W_d$, the function $\operatorname{context}(w, b)$ returns a sequence containing the following and preceding $b$ tokens from $W_d$ not including $w_i$ itself. The variable $b$ is called the window size.
  \begin{equation}
    \operatorname{context}(w_i,b)=(w_{i-b},\,\dots,\,w_{i-1},\,w_{i+1},\,\dots,\,w_{i+b})
  \end{equation}
\end{definition}

\begin{example}[Context window]
  A context window around the word ``next'' in the quote \eqref{eq:virginia-context} with a window size of 3,
  \begin{align}
    &\texttt{\fbox{We\vphantom{hy}} \fbox{know\vphantom{hy}} \fbox{not\vphantom{hy}} \fbox{what\vphantom{hy}} \fbox{comes\vphantom{hy}} \fbox{next\vphantom{hy}}, \fbox{or\vphantom{hy}} \fbox{what\vphantom{hy}} \fbox{follows\vphantom{hy}} \fbox{after\vphantom{hy}}}. \quad-\text{{\sffamily\small Virginia Woolf}}\nonumber\\[-.5em]
    &\texttt{~~~~~~~~~~~~-3~~~~-2~~~~-1~~~~~0~~~~+1~~~~+2~~~~+3}\label{eq:virginia-context}
  \end{align}
  is given by the sequence:\vspace{-0.5em}
  \begin{equation*}
    \operatorname{context}\big(\;\fbox{\small{\texttt{next}\vphantom{hy}}}\;, 3\big)= (\;\fbox{\small{\texttt{not}\vphantom{hy}}}\;, \;\fbox{\small{\texttt{what}\vphantom{hy}}}\;, \;\fbox{\small{\texttt{comes}\vphantom{hy}}}\;, \;\fbox{\small{\texttt{or}\vphantom{hy}}}\;, \;\fbox{\small{\texttt{what}\vphantom{hy}}}\;, \;\fbox{\small{\texttt{follows}\vphantom{hy}}}\;)
  \end{equation*}
  \vspace{0.5em}
\end{example}

\begin{definition}[Word-context co-occurrence count]
  Given a tokenized document $W_d$, we denote the number of times a token $c$ appears in the context of another token $w^*$: $\#_c\;\operatorname{context}(w^*)$. Then, we denote the total number of times $c$ appears in the context of any instance of the $w^*$ token throught the document $W_d$ by
  \begin{equation}
    f(w^*,c,W_d)=\sum_{\{w\in d\;\mid\; w=w^*\}}\#_c\;\operatorname{context}(w)
  \end{equation}
  and, given a corpus of documents $C$ we denote the sum of all these co-occurrence counts over all the documents in the corpus:
  \begin{equation}
    F(w^*,c)=\sum_{d\in C}f(w^*,c)
  \end{equation}
\end{definition}

From here it is straightforward to define the word-context co-occurrence matrix:

\begin{definition}[Word-context co-occurrence matrix]
  Given a corpus of documents $C=d_1,d_2,\dots,d_n$, and a tokenizer $T$, a term document matrix $\bf{X}$ is given by:
  \begin{equation}
    \bf{X}=
  \begin{bmatrix}
    F(w_1,c_1) & F(w_1,c_2) & \dots  & F(w_1,c_m) \\
    F(w_2,c_1) & F(w_2,c_2) & \dots  & F(w_2,c_m) \\
    \vdots        & \vdots        & \ddots & \vdots          \\
    F(w_m,c_1) & F(w_m,c_2) & \dots  & F(w_m,c_m) \\
  \end{bmatrix}
  \end{equation}
  where the $w_i$ and $c_i$ terms are are both from the vocabulary of $C$, $V(C)=w_1,w_2,\dots,w_m=c_1,c_2,\dots,\c_m$.
  The number of times the $i^{\text{th}}$ word appears in a context window with the $j^{\text{th}}$ context word is then given by $\bf{X}_{ij}$.
\end{definition}

\begin{example}[Meno's answer]
  A word-context co-occurrence matrix generated using this procedure with a window size of 10, using the works of Plato as its corpus identified these words as the ten \emph{most similar} (using the cosine measeure of similarity) to ``virtue'':
  \begin{center}{\footnotesize
      \captionsetup{width=.91\linewidth}
      \begin{tabular}{c r r r r r r r r r r}
        \toprule
        word & \multicolumn{1}{c}{virtue} & \multicolumn{1}{c}{wisdom} & \multicolumn{1}{c}{a} & \multicolumn{1}{c}{justice} & \multicolumn{1}{c}{this} & \multicolumn{1}{c}{an} & \multicolumn{1}{c}{men} & \multicolumn{1}{c}{courage} & \multicolumn{1}{c}{knowledge} & \multicolumn{1}{c}{all} \\
        \midrule
        similarity & 0.999 & 0.981 & 0.980 & 0.978 & 0.977 & 0.976 & 0.972 & 0.972 & 0.972 & 0.972 \\
        \bottomrule
      \end{tabular}
      \captionof{table}{The most similar words to ``virtue'' according to a word-context matrix trained on the works of Plato.}\label{tab:plato-doc-term}
    }\end{center}
  clearly stopwords like ``a'' and ``this'' have been included because of their high frequency in all contexts -this problem is discussed in \autoref{sec:weighting}.
\end{example}

The consequence of using a particular one these two main co-occurrence matrix types (with documents as the basis elements or with context words as basis elements) corresponds the difference in semantic relatedness discussed in \autoref{sec:semantic-relatedness}. Words that notably co-occur in the context of a document are likely to be \emph{paradigmatic paralells}, whereas words which co-occur in the context of a window around a target word are likely to be \emph{semantically similar}. \textcite{sahlgreen-2006-the-word-space-model} explores this connection in more depth in his thesis.

\section{Weighting}\label{sec:weighting}
The problem with raw co-occurrence counts is, like we see in \autoref{tab:plato-doc-term}, that ``stop words'' like \emph{the}, which tell us very little about the meaning of a given passage, are the highest weighted elements wheras domain specific terms which are the most informative only occur infrequently so contribute less to the overall position of a count-based vector.

We remedy this by applying a weighting function to the elements of $\bf{X}$. ``The idea of weighting is to give more weight to surprising events and less weight to expected events'' \parencite{turney10-from-frequen-to-meanin}.

For document-term matricies the most commonly used weighting method is the ``term frequency $\times$ inverse document frequency'' approach. The idea is that a term's weight sould be high if it was frequent in that document, but rare in other documents. This formulation has two parts: the term frequency and the inverse document frequency. Both of these parts have a variety of forms, often justified heuristically but (\textcite{robertson-2004-understanding-idf} includes a good discussion of justifications for the method -concluding that the theory of relevance weights \parencite{robertson-1976-relevance-weighting} makes the strongest cast) nevertheless these methods ``have proved extraudinarily robust and ddifficult to beat, even by much more carefully worked out models and theories'' \parencite{robertson-2004-understanding-idf}.

\begin{definition}[Term frequency]
  The term frequency should be higher if the term is very frequent in the document and lower otherwise.
  Given a corpus $C$, a tokenizer $T$, and a co-occurrence count function $f$ such that $f(w,W_d)$ equals the number of times that a token $w$ appears in a tokenized document $W_d$, the term frequency $\times$ inverse document frequency weighting for $w$ \& $d$ is given by:
  \begin{align}
    \operatorname{tf}(w,W_d)=\frac{f(w,W_d)}{\sum_{t\in W_d}f(t,W_d)}
  \end{align}
\end{definition}

\begin{definition}[Inverse document frequency]
  The inverse document frequency captures the idea of a 'rare' word: it should be higher if the term appears only in very few documents, and lower if the term appears in the majority of documents. First we define the document frequency:

  Given a corpus $C$, a tokenizer $T$, and a token $t$, the document frequency $\operatorname{df}$ of the term $t$ is given by:
  \begin{equation}
    \operatorname{df}_{t,d}=\frac{\#\{d\in C\;\mid\;t\in W_d\}}{\#C}
  \end{equation}
  The ``inverse document frequency'' $\operatorname{idf}$ is then given by: $\operatorname{idf}_t=\frac{1}{\operatorname{df}_t}$. This raw frequency is commonly scaled with a $\log$ function. The heuristic justification for this is because the significance of a term does not increase in proportion to its frequency -we want to emphasise the less frequent and ``squash'' the weights of the more frequent. Finally this gives us:
  \begin{equation}
    \operatorname{idf}_t=\log\frac{1}{\operatorname{df}}
  \end{equation}
\end{definition}

\begin{definition}[TF-IDF weighting]
  The final weight is then given by combining these terms:
  \begin{equation}
    \operatorname{tfidf}_{t,d}=\operatorname{tf}_{t,d}\operatorname{idf}_t
  \end{equation}
\end{definition}

For word-context matricies the most common approach is to use a pointwise-mutual-information weighting (PMI).

The concept of pointwise mutual information (PMI) is defined in terms two jointly distributed random variables $X$ and $Y$. In the case of word-context matricies we are interested in the joint distribution of two tokens throughout the text corpus, $ww$, and $c$. We apply the concept of pointwise mutual information by approximating the joint distribution probabilites of $w$ and $c$.

\begin{definition}[PMI weighting]
  Recall that the formula for PMI is:
  \begin{equation*}
    I(w,c)=\log\frac{P(w,c)}{P(w)P(c)}
  \end{equation*}
  Given a corpus $C$, a tokenizer $T$, and a word-context co-occurrence count function $F$ such that $F(w,c)$ equals the number of times $w$ appeared with $c$ in its context, approximate the value of the terms $P(w,c), P(w), P(c)$ as with the terms $p_{w,c}, p_w, p_c$ defined as follows:
  \begin{align}
    p_{w,c}&=\frac{F(w,c)}{\sum_{s\in V(C)}\sum_{t\in V(C)}F(t,s)}\\[0.5em]
    p_{w}&=\frac{\sum_{t\in V(C)}F(w,t)}{\sum_{s\in V(C)}\sum_{t\in V(C)}F(t,s)}\\[0.5em]
    p_{c}&=\frac{\sum_{t\in V(T)}F(c,t)}{\sum_{s\in V(C)}\sum_{t\in V(C)}F(t,s)}
  \end{align}
  Then the PMI weight is defined:
  \begin{equation}
    \operatorname{pmi} (w, c)=\log\frac{1 + p_{w,c}}{p_w p_c}
  \end{equation}
  The $+1$ within the logarithm is to avoid evaluating the undefined $\log 0$ since often the numerator $(p_{w,c})$ is equal to zero because the tokens $w$ and $c$ never co-occurred.
\end{definition}

\section{Dimensionality Reduction}
One main issue with the word spaces produced by count models is they are inconveniently high dimensional (as many dimensions as there are word-types in the vocabulary), and that they are \emph{sparse}: a large number of the entries are zero.

One common approach to tackle this problem is to factor the sparse matrix into smaller dense matricies using a technique called \emph{singular value decomposition} (SVD) \parencite{dumais-1988-using-lsa-to-improve}. Another approach is \emph{feature selection} where only those terms considered informative beyond a certain threshold are included in the co-occurrence matrix and the rest are discarded.

\begin{figure}[H]
  \centering
  \captionsetup{width=.91\linewidth}
  \includegraphics[scale=0.35]{figures/reduced-svd.png}
  \caption{Chart from \textcite{albright-2004-taming}. SVD decomposes $\bf{X}$ into three matricies: $\bf{U\sigma V}^T$. $\bf{U}$ and $\bf{V}$ are in column orthogonal form, and $\bf{\sigma}$ is a diagonal matrix of singular values. \parencite{golub13_matrix, turney10-from-frequen-to-meanin}}\label{fig:svd}
\end{figure}

\section{Comparison with other models}
Count models were the dominant approach to word embedding until \textcite{mikolov13-effic-estim-word-repres-vector-space} described a method for \emph{efficiently} generating embeddings using techniques from machine learning. Embeddings methods using machine learning techniques, called \emph{predict} models, initially developed somewhat independently. The first occurrence of this approach was in \parencite{bengio-2003-a-neural-prob-lang-model}, which described a machine-learning language model which produced word embeddings as a by-product of its primary function. Other predict models were concieved, improving upon this inital concept \parencite{morin-2005-hierarchical-probabilistic, mnih-2007-three-new-graphical-models, collobert-2008-a-unified-architecture} but this embedding paradigm remained less well known until until the success of \citeauthor{mikolov13-effic-estim-word-repres-vector-space}'s \parencite*{mikolov13-effic-estim-word-repres-vector-space} \texttt{word2vec} models encouraged direct comparisons and cross-paradigm research \parencite{baroni-etal-2014-dont, levy-2014-neural-WE-as}.

Although the methods seem quite different on the surface, \textcite{levy-2014-neural-WE-as} found that it is possible to view the operation of these predict models as \emph{implicitly} factoring a co-occurrence matrix -so theoretically the same information is being used by both models.

The methods we have presented here (document-term matricies and context-word matricies) are the foundation of the theory of count models but these particular models can vary greatly. Most notably, \citeauthor{pennington2014glove}'s \parencite*{pennington2014glove} GloVe model reports better results than other count-based models and prediction based models of the \texttt{word2vec} software package in the tasks of word-analogy and named entity recognition.

The cutting edge of embedding methods, at the time of writing is the ``deep contextualised model'' of \textcite{peters18-deep-contex-word-repres} which incorporates sentence-level information and leverages a language model to capture context-dependent aspects of word meaning. This is one particularly successful approach to tackling the difficulties endemic to word-meaning inference discussed in \nameref{chap:words} and \nameref{sec:semantic-relatedness}

%%% Local Variables:
%%% mode: latex
%%% TeX-master: "../main"
%%% End:

% ++ types of cooccurrence matrix
% ++ smoothing
% ++ dimension reduction
% ++ computation
% ++ comparison with other models
\chapter{Further Reading}\label{chap:further-reading}
\addcontentsline{toc}{chapter}{Further Reading}
In this section we briefly discuss practical considerations for the application of embedding algorithms and supply suggested resources for reference in each case.

\begin{infobox}{Selected topics and references}{\small
  \paragraph{Key references} These resources have been invaluable in writing this report:\hspace{1.4em}
  \begin{enumerate}
    \item On words: \emph{\citetitle{halliday-2004-lexicology, dixon02-word}} \parencite{halliday-2004-lexicology, dixon02-word},\vspace{-0.5em}
    \item On Count models: \emph{\citetitle{turney10-from-frequen-to-meanin, sahlgreen-2006-the-word-space-model}} \parencite{turney10-from-frequen-to-meanin, sahlgreen-2006-the-word-space-model},\vspace{-0.5em}, 
    \item An overview of embedding methods: \emph{\citetitle{wang-2019-evaluating}} \parencite{wang-2019-evaluating}
    \end{enumerate}
  \paragraph{Pre-processing}
  Although we discussed tokenization in this report we did not discuss the more general problem of text pre-processing. Purposes of pre-processing can be lemmatization (replacing words with their ``lemma'' form i.e., looks$\rightarrow$look and running$\rightarrow$run), lowercasing, stop-word removal, multiword grouping (identifying multi-part grammatical words as single tokens). Suggested reading:
  \begin{enumerate}
    \item \emph{\citetitle{uysal-2014-the-impact-of-preprocessing}} \parencite{uysal-2014-the-impact-of-preprocessing},\vspace{-0.5em}
    \item \emph{\citetitle{camacho-collados17-role-text-prepr-neural-networ-archit}} \parencite{camacho-collados17-role-text-prepr-neural-networ-archit}.
  \end{enumerate}
  \paragraph{Bias}\label{para:bias}
  Since embedding algorithms do not understand language, they just identify statistical regularities in corpora, if that training data contains bias then that bias will be represented (or even amplified) in the embedding output. Suggested reading:
  \begin{enumerate}
    \item \emph{\citetitle{caliskan-2017-semantics-derived}} \parencite{caliskan-2017-semantics-derived},\vspace{-0.5em}
    \item \emph{\citetitle{bolukbasi16-man-is-to-comput-progr}} \parencite{bolukbasi16-man-is-to-comput-progr}.
  \end{enumerate}
}\end{infobox}
% ++ preprocessing
% ++ bias
\chapter*{Conclusion}\label{chap:conclusion}
\addcontentsline{toc}{chapter}{Conclusion}
Word embeddings are an essential tool in the NLP toolbox -they are ``a standard component of most state-of-the-art NLP architectures'' \parencite{peters18-deep-contex-word-repres}. They enable many systems which have to interact with language in some capacity do so more `intellegently'. In this report we have tried to describe the problems which make the creation of \emph{truly} intellegent word embeddings more difficult. Engineering better methods for representing language requires understanding what written language \emph{is} and how it performs its function. Cutting edge methods like that of \textcite{devlin-etal-2019-bert}, which incorporate an element of language modelling, will likely become the standard for this reason. Despite this, it is remarkable how much semantic information is encoded by even the most simple models, applicable to any type of sequenced data because of their \emph{lack} of assumptions. The core idea behind word embeddings, \emph{the distributional hypthesis}, is a powerful one. Finally, for practitioners of these methods the \nameref{para:bias} aspect is essential to consider: embedded representations say nothing `objective' about the concepts they represent, they are only a reflection of their training data. This is not a shortcoming of the methods used, it should just inform their manner of application -a good use of this bias-capturing nature of embeddings is for studying bias itself, see \parencite{garg-2018-word-embeddings-quantify} for such an investigation.

\makebibliography
\end{document}
