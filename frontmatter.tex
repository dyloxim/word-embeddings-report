% general
\usepackage{fontspec}
\usepackage{xeCJK}
\usepackage{titlesec}
\usepackage{subfigure,epsfig,amsfonts}
\usepackage{amsmath}
\usepackage{amssymb}
\usepackage{mathrsfs}
\usepackage{wrapfig}
\newenvironment{figurewrap}
 {%
%  \setlength{\intextsep}{0pt}% <--- Wrong!
  \setlength{\columnsep}{15pt}%
  \wrapfloat{figure}%
 }
 {\endwrapfloat}
\usepackage{graphicx}
\usepackage{tikz,pgf}
\usetikzlibrary{backgrounds}
\usepackage{booktabs}
\usepackage{wasysym}
\usepackage{multicol}
\usepackage[linesnumbered,lined,boxed,commentsnumbered]{algorithm2e}
\usepackage{capt-of}
\usepackage{caption}
\usepackage{float}
\usepackage[dvipsnames]{xcolor}
\usepackage{soul}
\usepackage{empheq}
\usepackage{enumitem}
\usepackage[many]{tcolorbox}
\usepackage[
  left=1.35in,
  textwidth=5.9in,
  includeheadfoot,
  voffset=0.5em,
  headsep=1.0em,
  headheight=1.5em,
  textheight=635pt,
  footskip=3.0em
  ]{geometry}% http://ctan.org/pkg/geometry
\setlength{\intextsep}{1.5em}
\setlength{\abovedisplayskip}{-2.0em}
\setlength{\belowdisplayskip}{-2.0em}
\setlength{\abovedisplayshortskip}{-2.5em}
\setlength{\belowdisplayshortskip}{-2.5em}
\setlength{\belowcaptionskip}{0.0em}

% require: amsmath, xcolor, soul, empheq, [many]tcolorbox
% highglighting text
\definecolor{mygreen}{rgb}{0.51, 0.94, 0.63}
\newcommand{\hll}[1]{\colorbox{mygreen}{$\displaystyle #1$}}
\tcbset{
  highlight math style={
    colback=mygreen,
    arc=0pt,
    outer arc=0pt,
    boxrule=0pt,
    top=2pt,
    bottom=2pt,
    left=2pt,
    right=2pt,
  }
}

% encircling numbers
\newcommand*\circled[1]{\tikz[baseline=(char.base)]{
    \node[shape=circle,draw,inner sep=2pt] (char) {#1};}}

% for embedding svg figures made with inkscape
\usepackage{svg}
% frames
\usepackage{framed}

\makeatletter
\newcommand\notsotiny{\@setfontsize\notsotiny\@vipt\@viipt}
\makeatother

\titlespacing{\chapter}{0pt}{-80pt}{1cm}
\titlespacing{\paragraph}{0pt}{0em}{0.8em}

% theorem environments
\usepackage{amsthm, thmtools}

% headers
\usepackage{fancyhdr}
\pagestyle{fancy}
\fancyhead[L]{\rightmark}
\fancyhead[R]{\thepage}
\renewcommand{\headrulewidth}{0pt}

%% subsubsections
\setcounter{secnumdepth}{3} % how many sectioning levels to assign numbers to
\setcounter{tocdepth}{3}    % how many sectioning levels to show in ToC
\newcounter{subsubsection}[subsection]

% references
\usepackage[colorlinks = true,%
linkcolor = black,%
urlcolor  = black,%
citecolor = black,%
anchorcolor = black]{hyperref}
\usepackage{cleveref}

% Bibliography
\usepackage[
    backend    =      biber,%
    style      = authoryear,%
    sortlocale =      en_GB,%
    url        =      false,% 
    doi        =       true,%
    eprint     =      false]{biblatex}
\addbibresource{refs.bib}

% glossary functions
% \usepackage[toc]{glossaries}
% uncomment to make glossary words bold
% \renewcommand*{\glstextformat}[1]{\textbf{#1}}

\newfontfamily\quotefont[ligatures=TeX]{Optima}
\DeclareTextFontCommand{\textquotefont}{\normalfont\quotefont}
\newcommand{\authorquote}[2]{
\begin{center}\begin{quote}
\parbox{0.25cm}{\Large{``}}{\small{\textquotefont{{#1}}}}$\,$\parbox{0.25cm}{\Large{''}} \vspace{-1.30em}\flushright{\small{\textquotefont{{#2}}}}
\end{quote}\end{center}
\vspace{-0.60em}
}
\newcommand{\basicquote}[1]{
\begin{center}\begin{quote}
\parbox{0.25cm}{\Large{``}}{\small{\textquotefont{{#1}}}}$\,$\parbox{0.25cm}{\Large{''}}
\end{quote}\end{center}
}

\usepackage[framemethod=TikZ]{mdframed}
\usetikzlibrary{calc}

\makeatletter
\newrobustcmd*\mdf@tikzbox@tfl@spare[1]{%three or four borders
    \clip(0,0)rectangle(\mdfboundingboxwidth,\mdfboundingboxheight);%
    \begin{scope}[mdfcorners]%
       \clip[preaction=mdfouterline]%
            [postaction=mdfbackground]%
            [postaction=mdfinnerline]#1;%
    \end{scope}%
    \path[mdfmiddleline,mdfcorners] ($(O|-P)-(0,0.25cm)$)--(O|-P)--(P)--($(P)-(0,0.25cm)$);
    \path[mdfmiddleline,mdfcorners] ($(P|-O)+(0,0.25cm)$)--(P|-O)--(O)--($(O)+(0,0.25cm)$);
  }%
\newrobustcmd*\changelinestyle{\let\mdf@tikzbox@tfl\mdf@tikzbox@tfl@spare}
\makeatother


% theorem styles
\declaretheoremstyle[%
headfont=\sffamily\bfseries,%
notefont=\mdseries\itshape\sffamily,%
notebraces={}{},%
headpunct=,%
bodyfont=\sffamily,%
headformat=\NAME~\NUMBER~--\NOTE\hfill\smallskip\linebreak,%
preheadhook=\begin{leftbar},%
  postfoothook=\end{leftbar},%
]{customDefinition}
\declaretheoremstyle[%
  spaceabove=2pt, spacebelow=5pt,
  headfont=\bfseries,
  postheadspace=1.4em,
  notefont=\normalfont\itshape, notebraces={(}{)},
  bodyfont=\normalfont,
]{example}

% theorem environments
\declaretheorem[name=Definition,
style=customDefinition,
numberwithi=chapter,
refname={definition,definitions},
Refname={Definition,Definitions}]{definition}
\declaretheorem[name=Note,
style=definition,
numberwithin=chapter,
refname={note,notes},
Refname={Note,Notes}]{note}
\declaretheorem[name=Example,
style=example,
numberwithin=chapter,
refname={example,examples},
Refname={Example,Examples}]{example}
\declaretheorem[name=Example,
style=example,
numberwithin=chapter,
refname={example,examples},
Refname={Example,Examples}]{example*}
\definecolor{mygrey}{rgb}{0.92, 0.92, 0.92}
\mdfdefinestyle{infobox}{%
  backgroundcolor=white,
  roundcorner=0.7pt,%
  middlelinewidth=0.7pt,%
  skipabove=10pt,%
  splittopskip=1.5em,%
  firstextra={\draw[dashed,line width=0.7pt,xshift=0.7pt] (O) -- (P|-O);},%
  secondextra={\draw[dashed,line width=0.7pt,xshift=-0.7pt] (O|-P) -- (P);},%
  middleextra={\draw[dashed,line width=0.7pt,xshift=0.7pt] (O) -- (P|-O);\draw[dashed,line width=0.7pt,xshift=-0.7pt] (O|-P) -- (P);},%
}

\newenvironment{infobox}[1]%
{\begin{mdframed}[%
    style=infobox,%
    frametitle={\mdseries\sffamily Summary: \vphantom{$frac{1}{2}$}#1},%
    skipabove=\baselineskip plus 3pt minus 1pt,%
    skipbelow=\baselineskip plus 3pt minus 1pt,%
    linewidth=0.5pt,%
    frametitlerule=true,%
    frametitlebackgroundcolor=mygrey]
  }
  {\end{mdframed}}

\surroundwithmdframed[settings=%
\changelinestyle,%
middlelinecolor=black,%
roundcorner=0.7pt,%
middlelinewidth=0.7pt,%
skipabove=10pt,%
splittopskip=1.5em,%
  firstextra={\draw[white, dashed,line width=1pt,xshift=0.7pt, pattern= on 3pt off 3pt] (O) -- (P|-O);},%
  secondextra={\draw[white, dashed,line width=1pt,xshift=-0.7pt, pattern= on 3pt off 3pt] (O|-P) -- (P);},%
  middleextra={\draw[white, dashed,line width=1pt,xshift=0.7pt, pattern= on 3pt off 3pt] (O) -- (P|-O);\draw[dashed,line width=0.7pt,xshift=-0.7pt] (O|-P) -- (P);},%
]{example}
\surroundwithmdframed[settings=%
\changelinestyle,%
middlelinecolor=black,%
roundcorner=0.7pt,%
middlelinewidth=0.7pt,%
skipabove=10pt,%
splittopskip=1.5em,%
  firstextra={\draw[white, dashed,line width=1pt,xshift=0.7pt, pattern= on 3pt off 3pt] (O) -- (P|-O);},%
  secondextra={\draw[white, dashed,line width=1pt,xshift=-0.7pt, pattern= on 3pt off 3pt] (O|-P) -- (P);},%
  middleextra={\draw[white, dashed,line width=1pt,xshift=0.7pt, pattern= on 3pt off 3pt] (O) -- (P|-O);\draw[dashed,line width=0.7pt,xshift=-0.7pt] (O|-P) -- (P);},%
]{example*}

\renewenvironment{leftbar}{%
  \def\FrameCommand{{\vrule width 3pt} \hspace{10pt}}%
  \MakeFramed {\advance\hsize-\width \FrameRestore}}%
{\endMakeFramed}
