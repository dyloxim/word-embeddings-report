% general
\usepackage{fontspec}
\usepackage{xeCJK}
\usepackage{titlesec}
\usepackage{subfigure,epsfig,amsfonts}
\usepackage{amsmath}
\usepackage{amssymb}
\usepackage{mathrsfs}
\usepackage{wrapfig}
\usepackage{graphicx}
\usepackage{tikz,pgf}
\usepackage{float}
\usepackage{booktabs}
\usepackage{wasysym}
\usepackage{multicol}
\usepackage[linesnumbered,lined,boxed,commentsnumbered]{algorithm2e}
\usepackage[
  top=0.9in,
  bottom=0.8in,
  includeheadfoot,
  headsep=0.2in,
  left=1.35in,
  textwidth=5.9in,
  headheight=0.0in]{geometry}% http://ctan.org/pkg/geometry
\titlespacing{\chapter}{0pt}{-80pt}{1cm}% <-- CHANGE DONE HERE!!

% theorem environments
\usepackage{amsthm, thmtools}

% headers
\usepackage{fancyhdr}
\pagestyle{fancy}
\fancyhf{}
\fancyhead[L]{\rightmark}
\fancyhead[R]{\thepage}
\renewcommand{\headrulewidth}{0pt}

%% subsubsections
\setcounter{secnumdepth}{4} % how many sectioning levels to assign numbers to
\setcounter{tocdepth}{4}    % how many sectioning levels to show in ToC
\newcounter{subsubsection}[subsection]

% references
\usepackage{hyperref, cleveref}

% Bibliography
\usepackage[
    backend=biber,
    style=authoryear,
    sortlocale=en_GB,
    url=false, 
    doi=true,
    eprint=false
]{biblatex}
\addbibresource{refs.bib}

% glossary functions
% \usepackage[toc]{glossaries}
% uncomment to make glossary words bold
% \renewcommand*{\glstextformat}[1]{\textbf{#1}}

% theorem styles
 \declaretheoremstyle[headfont=\sffamily\bfseries,%
 notefont=\rmfamily\itshape,%
 notebraces={}{},%
 headpunct=,%
 bodyfont=\sffamily,%
 headformat=\NAME~\NUMBER~--\NOTE\hfill\smallskip\linebreak,%
 preheadhook=\begin{leftbar},%
 postfoothook=\end{leftbar},%
 ]{customDefinition}
\declaretheoremstyle[
  spaceabove=6pt, spacebelow=6pt,
  headfont=\normalfont\itshape,
  notefont=\mdseries, notebraces={(}{)},
  bodyfont=\normalfont,
  postheadspace=1em
]{example}

% theorem environments
\declaretheorem[name=Definition,
  style=customDefinition,
  numberwithin=section,
  refname={definition,definitions},
  Refname={Definition,Definitions}]{definition}
\declaretheorem[name=Note,
  style=definition,
  numberwithin=section,
  refname={note,notes},
  Refname={Note,Notes}]{note}
\declaretheorem[name=Example,
  style=example,
  numberwithin=section,
  refname={example,examples},
  Refname={Example,Examples}]{example}

% for embedding svg figures made with inkscape
\usepackage{svg}

% frames
\usepackage[svgnames]{xcolor}
\usepackage{framed}
\renewenvironment{leftbar}{%
  \def\FrameCommand{{\vrule width 3pt} \hspace{10pt}}%
  \MakeFramed {\advance\hsize-\width \FrameRestore}}%
{\endMakeFramed}
